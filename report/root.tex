\documentclass[11pt,a4paper]{article}

% ==================== PACKAGES ====================
\usepackage[utf8]{inputenc}
\usepackage[english]{babel}
\usepackage{amsmath}
\usepackage{graphicx}
\usepackage{float}
\usepackage{caption}
\usepackage{hyperref}
\usepackage{xcolor}
\usepackage{geometry}
\usepackage{booktabs}
\usepackage{siunitx}
\usepackage{listings}

% ==================== CUSTOM COMMANDS ====================
\newcommand{\TODO}[1]{\textcolor{red}{\textbf{TODO: #1}}}

% ==================== LISTINGS SETUP ====================
\lstset{
    language=Matlab,
    basicstyle=\ttfamily\footnotesize,
    keywordstyle=\color{blue},
    commentstyle=\color{green!60!black},
    stringstyle=\color{red},
    numbers=left,
    numberstyle=\tiny\color{gray},
    stepnumber=1,
    numbersep=5pt,
    backgroundcolor=\color{white},
    showspaces=false,
    showstringspaces=false,
    showtabs=false,
    frame=single,
    tabsize=2,
    captionpos=b,
    breaklines=true,
    breakatwhitespace=false,
    escapeinside={\%*}{*)},
    morekeywords={function, end, if, else, for, while, switch, case, otherwise}
}

% ==================== PAGE SETUP ====================
\geometry{
    a4paper,
    left=2.5cm,
    right=2.5cm,
    top=2.5cm,
    bottom=2.5cm
}

% ==================== HYPERLINKS ====================
\hypersetup{
    colorlinks=true,
    linkcolor=blue,
    urlcolor=cyan,
    citecolor=blue,
    pdftitle={Power Electronics I - Course Project},
    pdfauthor={Azab, Saber, Abdelhafez}
}

% ==================== GRAPHICS PATH ====================
\graphicspath{{figures/}}

% ==================== DOCUMENT BEGIN ====================
\begin{document}

% ==================== CUSTOM TITLE PAGE ====================
\begin{titlepage}
    \centering
    \vspace*{2cm}
    
    {\huge\bfseries Power Electronics I\par}
    \vspace{1cm}
    {\Large Course Project Report\par}
    \vspace{0.5cm}
    {\large Controlled Rectifiers for Battery Charging\par}
    
    \vspace{3cm}
    
    {\Large\textbf{Team Members}\par}
    \vspace{0.5cm}
    {\large
        Mohammed Azab\par
        \vspace{0.3cm}
        Basant Saber\par
        \vspace{0.3cm}
        Habiba Abdelhafez\par
    }
    
    \vfill
    
    {\large\textbf{German International University in Berlin}\par}
    {\large Faculty of Engineering\par}
    
    \vspace{1cm}
    
    {\large\textbf{Instructor:} Dr.-Ing. Moustafa Adly\par}
    
    \vspace{1cm}
    
    {\large Fall Semester 2025\par}
    {\large November 25, 2025\par}
    
\end{titlepage}

% Table of Contents
\tableofcontents
\newpage

% ==================== ABSTRACT ====================
\section*{Abstract}
\addcontentsline{toc}{section}{Abstract}

This project presents the design and analysis of controlled rectifier circuits using thyristors (SCRs) for battery charging applications. Three different rectifier configurations are implemented and simulated in MATLAB: half-wave controlled rectifier, full-wave center-tapped rectifier, and full-wave bridge rectifier. The study examines the relationship between firing angle and charging characteristics, including average and RMS voltage and current values. Performance comparisons demonstrate that full-wave configurations provide superior charging performance with reduced charging times and improved efficiency compared to half-wave rectifiers. The analysis includes circuit modeling, waveform generation, and comprehensive performance metrics for each topology.

% ==================== MAIN CONTENT ====================

% ==================== PART I ====================
\section{Introduction}

\subsection{Objectives}

The aim of this project is to design and simulate controlled rectifiers using thyristors (SCRs) for battery charging applications. A rectifier converts AC signals into DC signals by blocking the negative half of the AC waveform, making the output unidirectional for proper battery charging.

Three rectifier configurations were developed in MATLAB:
\begin{enumerate}
    \item \textbf{Half-wave Controlled Rectifier}: Single thyristor allowing only positive half-cycles
    \item \textbf{Full-wave Center-Tapped Rectifier}: Center-tapped transformer with two thyristors
    \item \textbf{Full-wave Bridge Rectifier}: Four thyristors in bridge configuration
\end{enumerate}

The analysis examines the relationship between firing angle and charging characteristics (average and RMS voltage, current, charging time) to compare the performance of each configuration.

\subsection{Background}

For safe and efficient operation, batteries require stable DC current with controlled voltage and current. Silicon-Controlled Rectifiers (SCRs) regulate power precisely, preventing overcharging and ensuring long battery life. SCRs can handle high currents and powers while adapting to battery state of charge (SoC) and temperature.

\subsection{Scope}

This project covers:
\begin{itemize}
    \item Analytical modeling using MATLAB
    \item Three rectifier configurations
    \item Performance comparison based on firing angle
    \item Power loss analysis including thyristor non-idealities
    \item State of Charge tracking
\end{itemize}

\section{Design and Implementation}

\subsection{MATLAB Function Architecture}

Each rectifier type is implemented as a separate MATLAB function:
\begin{itemize}
    \item \texttt{half\_wave\_charger.m} - Half-wave controlled rectifier
    \item \texttt{full\_wave\_ct\_charger.m} - Full-wave center-tapped rectifier  
    \item \texttt{full\_wave\_bridge\_charger.m} - Full-wave bridge rectifier
    \item \texttt{params.m} - System parameter configuration
\end{itemize}

All functions share common features:
\begin{enumerate}
    \item Input parameter handling for battery, supply, and thyristor specifications
    \item Firing angle sweep from 0° to 180°
    \item Calculation of voltage, current, and charging metrics
    \item Power loss analysis (battery, thyristor conduction, blocking, switching)
    \item State of Charge tracking
    \item Comprehensive visualization and data output
\end{enumerate}

\subsection{Half-Wave Controlled Rectifier}

Uses a single thyristor that conducts only during positive half-cycles when:
\begin{itemize}
    \item Thyristor is forward-biased
    \item Gate trigger is applied ($\theta \geq \alpha$)
    \item Output voltage exceeds battery EMF
\end{itemize}

\textbf{Key Equations:}
\begin{equation}
    V_{dc} = \frac{V_m}{2\pi}(1 + \cos\alpha) - V_t
\end{equation}
\begin{equation}
    I_{avg} = \frac{V_{dc} - V_{bat}}{R_{bat}}
\end{equation}
\begin{equation}
    t_{charge} = \frac{(SoC_{target} - SoC_{init}) \cdot C_{Ah} \cdot 3600}{I_{avg}}
\end{equation}

\subsection{Full-Wave Center-Tapped Rectifier}

Employs center-tapped transformer with two thyristors alternating each half-cycle, producing two output pulses per AC cycle.

\textbf{Advantages:} Higher average voltage, reduced ripple, faster charging

\textbf{Key Equation:}
\begin{equation}
    V_{dc} = \frac{2V_m}{\pi}(1 + \cos\alpha) - V_t
\end{equation}

\subsection{Full-Wave Bridge Rectifier}

Four thyristors in bridge arrangement. T1-T3 conduct during positive half-cycle, T2-T4 during negative half-cycle.

\textbf{Advantages:} Standard transformer (no center-tap), better transformer utilization

\textbf{Trade-off:} Four thyristors vs two thyristor drops in series

\textbf{Key Equation:}
\begin{equation}
    V_{dc} = \frac{2V_m}{\pi}(1 + \cos\alpha) - 2V_t
\end{equation}

\subsection{Power Loss Modeling}

Battery internal losses:
\begin{equation}
    P_{batt} = I_{rms}^2 \cdot R_{bat}
\end{equation}

Thyristor conduction losses:
\begin{equation}
    P_{th,cond} = V_t \cdot I_{avg} + R_{th} \cdot I_{rms}^2
\end{equation}

Thyristor blocking losses:
\begin{equation}
    P_{block} = V_{block,avg} \cdot I_{leak}
\end{equation}

Switching losses:
\begin{equation}
    P_{switch} = f \cdot (E_{on} + E_{off})
\end{equation}
where $E_{on} = \frac{1}{6}V_{block}I_{peak}t_{rise}$ and $E_{off} = \frac{1}{6}V_{block}I_{peak}t_{fall}$.

\subsection{State of Charge Tracking}

\begin{equation}
    SoC(t) = SoC_0 + \frac{100}{Q_{capacity}} \int_0^t I_{charge}(\tau) d\tau
\end{equation}
where $Q_{capacity}$ is in Coulombs (Ah × 3600).

\section{Part I: Results and Analysis}

\subsection{Half-Wave Rectifier Results}

\subsubsection{Firing Angle vs. Charging Time}
\TODO{Insert your main result plot}

\begin{figure}[H]
    \centering
    % \includegraphics[width=0.8\textwidth]{half_wave_alpha_vs_time.png}
    \caption{Firing angle vs. charging time for half-wave rectifier}
    \label{fig:half_wave_results}
\end{figure}

\TODO{Describe the plot:}
\begin{itemize}
    \item Trend observed (increasing $\alpha$ increases charging time)
    \item Physical explanation
    \item Key observations from the curve
\end{itemize}

\subsubsection{Output Voltage and Current Waveforms}
\TODO{Show representative waveforms for different firing angles}

\begin{figure}[H]
    \centering
    \begin{subfigure}[b]{0.45\textwidth}
        % \includegraphics[width=\textwidth]{half_wave_alpha_30.png}
        \caption{$\alpha = 30°$}
    \end{subfigure}
    \hfill
    \begin{subfigure}[b]{0.45\textwidth}
        % \includegraphics[width=\textwidth]{half_wave_alpha_90.png}
        \caption{$\alpha = 90°$}
    \end{subfigure}
    \caption{Output voltage and current waveforms for half-wave rectifier}
    \label{fig:half_wave_waveforms}
\end{figure}

\subsubsection{Performance Metrics}
\TODO{Create table with key metrics:}

\begin{table}[H]
\centering
\caption{Half-Wave Rectifier Performance Metrics}
\label{tab:half_wave_metrics}
\begin{tabular}{@{}lllll@{}}
\toprule
\textbf{Firing Angle} & \textbf{$V_{dc}$ (V)} & \textbf{$I_{dc}$ (A)} & \textbf{Charging Time (h)} & \textbf{Efficiency (\%)} \\ \midrule
$0°$ & \TODO{} & \TODO{} & \TODO{} & \TODO{} \\
$30°$ & \TODO{} & \TODO{} & \TODO{} & \TODO{} \\
$60°$ & \TODO{} & \TODO{} & \TODO{} & \TODO{} \\
$90°$ & \TODO{} & \TODO{} & \TODO{} & \TODO{} \\
$120°$ & \TODO{} & \TODO{} & \TODO{} & \TODO{} \\ \bottomrule
\end{tabular}
\end{table}

\subsection{Full-Wave Center-Tapped Rectifier Results}

\subsubsection{Firing Angle vs. Charging Time}
\TODO{Insert your main result plot}

\begin{figure}[H]
    \centering
    % \includegraphics[width=0.8\textwidth]{full_wave_ct_alpha_vs_time.png}
    \caption{Firing angle vs. charging time for center-tapped rectifier}
    \label{fig:full_wave_ct_results}
\end{figure}

\subsubsection{Output Voltage and Current Waveforms}
\TODO{Show representative waveforms}

\subsubsection{Performance Metrics}
\TODO{Create similar table as above}

\begin{table}[H]
\centering
\caption{Full-Wave Center-Tapped Rectifier Performance Metrics}
\label{tab:full_wave_ct_metrics}
\begin{tabular}{@{}lllll@{}}
\toprule
\textbf{Firing Angle} & \textbf{$V_{dc}$ (V)} & \textbf{$I_{dc}$ (A)} & \textbf{Charging Time (h)} & \textbf{Efficiency (\%)} \\ \midrule
$0°$ & \TODO{} & \TODO{} & \TODO{} & \TODO{} \\
$30°$ & \TODO{} & \TODO{} & \TODO{} & \TODO{} \\
$60°$ & \TODO{} & \TODO{} & \TODO{} & \TODO{} \\
$90°$ & \TODO{} & \TODO{} & \TODO{} & \TODO{} \\
$120°$ & \TODO{} & \TODO{} & \TODO{} & \TODO{} \\ \bottomrule
\end{tabular}
\end{table}

\subsection{Full-Wave Bridge Rectifier Results}

\subsubsection{Firing Angle vs. Charging Time}
\TODO{Insert your main result plot}

\begin{figure}[H]
    \centering
    % \includegraphics[width=0.8\textwidth]{full_wave_bridge_alpha_vs_time.png}
    \caption{Firing angle vs. charging time for bridge rectifier}
    \label{fig:full_wave_bridge_results}
\end{figure}

\subsubsection{Output Voltage and Current Waveforms}
\TODO{Show representative waveforms}

\subsubsection{Performance Metrics}
\TODO{Create similar table as above}

\subsection{Comparative Analysis}

\subsubsection{Side-by-Side Comparison}
\TODO{Create comparative plot with all three configurations}

\begin{figure}[H]
    \centering
    % \includegraphics[width=0.8\textwidth]{comparison_all_configs.png}
    \caption{Comparison of firing angle vs. charging time for all configurations}
    \label{fig:comparison_all}
\end{figure}

\subsubsection{Performance Comparison Table}
\TODO{Compare all three at same firing angle}

\begin{table}[H]
\centering
\caption{Comparison of Rectifier Configurations at $\alpha = 60°$}
\label{tab:comparison}
\begin{tabular}{@{}llllll@{}}
\toprule
\textbf{Configuration} & \textbf{$V_{dc}$ (V)} & \textbf{$I_{dc}$ (A)} & \textbf{Time (h)} & \textbf{$\eta$ (\%)} & \textbf{Ripple} \\ \midrule
Half-Wave & \TODO{} & \TODO{} & \TODO{} & \TODO{} & \TODO{High} \\
Full-Wave CT & \TODO{} & \TODO{} & \TODO{} & \TODO{} & \TODO{Medium} \\
Full-Wave Bridge & \TODO{} & \TODO{} & \TODO{} & \TODO{} & \TODO{Medium} \\ \bottomrule
\end{tabular}
\end{table}

\subsection{Optional Results}

\subsubsection{State of Charge Progression}
\TODO{If implemented, show SoC vs. time plots}

\begin{figure}[H]
    \centering
    % \includegraphics[width=0.8\textwidth]{soc_vs_time.png}
    \caption{State of Charge progression during charging}
    \label{fig:soc_progression}
\end{figure}

\subsubsection{Power Loss Analysis}
\TODO{If implemented, show power loss calculations}

\begin{table}[H]
\centering
\caption{Average Power Losses in Different Configurations}
\label{tab:power_losses}
\begin{tabular}{@{}lllll@{}}
\toprule
\textbf{Configuration} & \textbf{Conduction Loss (W)} & \textbf{Switching Loss (W)} & \textbf{Total Loss (W)} & \textbf{Efficiency (\%)} \\ \midrule
Half-Wave & \TODO{} & \TODO{} & \TODO{} & \TODO{} \\
Full-Wave CT & \TODO{} & \TODO{} & \TODO{} & \TODO{} \\
Full-Wave Bridge & \TODO{} & \TODO{} & \TODO{} & \TODO{} \\ \bottomrule
\end{tabular}
\end{table}

\subsection{Observations}
\TODO{List key observations from your results:}
\begin{enumerate}
    \item Effect of firing angle on charging time
    \item Comparison of half-wave vs. full-wave performance
    \item Difference between center-tapped and bridge configurations
    \item Impact of thyristor non-idealities (if modeled)
    \item Practical considerations for each configuration
\end{enumerate}

\section{Discussion}

\subsection{Analysis of Results}

\subsubsection{Firing Angle Effect}
Increasing firing angle significantly increases charging time because:
\begin{itemize}
    \item Delays thyristor turn-on, reducing conduction period
    \item Average voltage decreases: $V_{dc} = \frac{V_m}{k\pi}(1+\cos\alpha)$ where $k=2$ (half-wave), $k=1$ (full-wave)
    \item Lower voltage reduces charging current: $I_{avg} = (V_{dc} - V_{bat})/R_{bat}$
    \item Relationship is highly nonlinear with exponential growth as $\alpha \to 180°$
\end{itemize}

Practical firing angles: $0^\circ$ to $90^\circ$. Beyond $90^\circ$, output voltage too low for effective charging.

\subsubsection{Half-Wave vs Full-Wave}
Full-wave rectifiers are clearly superior:
\begin{itemize}
    \item 50\% reduction in charging time (two pulses per cycle)
    \item Double ripple frequency (100Hz vs 50Hz) - easier filtering
    \item Better transformer utilization
    \item Lower harmonic content
\end{itemize}

Half-wave only justified for very low-power, cost-sensitive applications.

\subsubsection{Center-Tapped vs Bridge}

\textbf{Center-Tapped:}
\begin{itemize}
    \item Pros: 2 thyristors, one $V_t$ drop, simpler gate drive
    \item Cons: Requires center-tapped transformer, poor transformer utilization
\end{itemize}

\textbf{Bridge:}
\begin{itemize}
    \item Pros: Standard transformer, better utilization, widely available
    \item Cons: 4 thyristors, two $V_t$ drops, complex gate drive
\end{itemize}

\textbf{Selection:} Use bridge for most applications (standard transformers). Use center-tapped if already available or when minimizing voltage drops is critical.

\subsection{Practical Considerations}

SoC tracking enables adaptive control: fast charge at low SoC, taper to trickle at high SoC, prevents overcharging.

Power loss analysis shows battery internal resistance dominates at high currents. Trade-off between fast charging (higher losses) and efficiency.

\subsection{Design Recommendations}

\begin{enumerate}
    \item \textbf{Industrial Applications}: Use full-wave bridge, operate at $\alpha = 15^\circ$--$45^\circ$, implement SoC-based control
    \item \textbf{Cost-Sensitive}: Consider half-wave if time not critical, optimize for balance of performance and stress
    \item \textbf{High-Efficiency}: Select low $V_t$ thyristors, optimize firing angle, consider active cooling for high power
\end{enumerate}

\subsection{Practical Considerations}

\subsubsection{Battery Charging Profile}
\TODO{Discuss:}
\begin{itemize}
    \item Constant current vs. constant voltage charging
    \item Impact of battery chemistry (lead-acid, Li-ion, etc.)
    \item Charging stages (bulk, absorption, float)
    \item How controlled rectifiers can implement different charging profiles
\end{itemize}

\subsubsection{Thyristor Selection}
\TODO{Discuss practical selection criteria:}
\begin{itemize}
    \item Voltage rating requirements
    \item Current rating requirements
    \item $dv/dt$ and $di/dt$ ratings
    \item Gate trigger requirements
    \item Thermal considerations
\end{itemize}


\subsection{Limitations of the Analysis}

\subsubsection{Model Simplifications}
\TODO{Acknowledge limitations:}
\begin{itemize}
    \item Simplified battery model (constant EMF and resistance)
    \item Ideal or simplified thyristor characteristics
    \item Neglected parasitic elements
    \item Steady-state analysis only
    \item No closed-loop control consideration
\end{itemize}

\subsubsection{Real-World Factors Not Considered}
\TODO{List factors that would affect real implementation:}
\begin{itemize}
    \item Supply voltage variations and harmonics
    \item Temperature effects on battery and components
    \item Component tolerances
    \item EMI/EMC considerations
    \item Aging effects
\end{itemize}

\subsection{Design Trade-offs}

\subsubsection{Cost vs. Performance}
\TODO{Analyze:}
\begin{itemize}
    \item Component count and cost
    \item Transformer requirements and cost
    \item Control circuit complexity
    \item Overall system cost for each configuration
\end{itemize}

\subsubsection{Efficiency vs. Control Range}
\TODO{Discuss:}
\begin{itemize}
    \item Efficiency variation with firing angle
    \item Optimal operating range
    \item Trade-off between controllability and efficiency
\end{itemize}

\subsubsection{Ripple vs. Complexity}
\TODO{Analyze:}
\begin{itemize}
    \item Output voltage ripple characteristics
    \item Need for filtering
    \item Impact on battery life
    \item Filter design considerations
\end{itemize}

\subsection{Recommendations}

\subsubsection{Configuration Selection Guidelines}
\TODO{Provide recommendations based on your analysis:}

\paragraph{Use Half-Wave When:}
\begin{itemize}
    \item \TODO{e.g., Very low power applications}
    \item \TODO{Cost is critical}
    \item \TODO{Charging time is not critical}
\end{itemize}

\paragraph{Use Center-Tapped When:}
\begin{itemize}
    \item \TODO{e.g., Center-tapped transformer already available}
    \item \TODO{Simple control is desired}
    \item \TODO{Medium power applications}
\end{itemize}

\paragraph{Use Bridge When:}
\begin{itemize}
    \item \TODO{e.g., Most general-purpose applications}
    \item \TODO{Standard transformer available}
    \item \TODO{Higher efficiency needed}
\end{itemize}

\subsubsection{Operating Point Recommendations}
\TODO{Suggest optimal operating conditions:}
\begin{itemize}
    \item Recommended firing angle range for each configuration
    \item Charging current limits
    \item Termination criteria
\end{itemize}

\subsection{Future Improvements}

\subsubsection{Model Enhancements}
\TODO{Suggest improvements to the model:}
\begin{itemize}
    \item More sophisticated battery model (e.g., equivalent circuit model)
    \item Detailed thyristor switching model
    \item Temperature dependence
    \item Closed-loop control simulation
\end{itemize}

\subsubsection{Analysis Extensions}
\TODO{Suggest additional analyses:}
\begin{itemize}
    \item Harmonic analysis
    \item Power factor analysis
    \item EMI prediction
    \item Reliability analysis
    \item Cost-benefit analysis
\end{itemize}


% ==================== PART II ====================
\section{Part II: Load Analysis - Introduction}

\subsection{Objectives}
\TODO{Expand on the following objectives:}

The primary objectives of Part II are:

\begin{itemize}
    \item Simulate and analyze controlled rectifiers under various load conditions
    \item Study the behavior of rectifiers with three load types:
    \begin{itemize}
        \item Resistive load (R)
        \item Resistive-inductive load (RL)
        \item Highly inductive load (L >> R)
    \end{itemize}
    \item Investigate current continuity and discontinuity modes
    \item Analyze the role of free-wheeling diodes
    \item Design firing circuits using pulse generators
    \item Compare half-wave and full-wave configurations
    \item Study the phenomenon of negative output voltage with positive current
\end{itemize}

\subsection{Background}
\TODO{Provide background on:}
\begin{itemize}
    \item Why load type matters in rectifier operation
    \item Inductive load effects on current waveforms
    \item Applications of controlled rectifiers with different loads:
    \begin{itemize}
        \item DC motor drives (RL load)
        \item Battery charging (EMF load - covered in Part I)
        \item Electrochemical processes
        \item High-power DC supplies
    \end{itemize}
\end{itemize}

\subsection{Problem Statement}
\TODO{State the specific problems addressed:}
\begin{itemize}
    \item Understanding current continuity conditions
    \item Predicting output characteristics for different loads
    \item Determining when free-wheeling diodes are beneficial
    \item Analyzing power flow and four-quadrant operation
\end{itemize}

\subsection{Simulation vs. Analytical Approach}
\TODO{Explain why Simulink is used for Part II:}
\begin{itemize}
    \item Complex waveforms difficult to analyze analytically
    \item Transient behavior important for inductive loads
    \item Visual representation of voltage and current waveforms
    \item Easy parameter variation and comparison
    \item Validation of analytical calculations
\end{itemize}

\subsection{Scope}
This part covers:
\begin{itemize}
    \item Simulink modeling and simulation
    \item MATLAB scripts for analytical calculations
    \item Systematic comparison of configurations
    \item Firing circuit design
\end{itemize}

This part does not cover:
\begin{itemize}
    \item Detailed component-level modeling
    \item Closed-loop control design
    \item Hardware implementation
\end{itemize}

\section{Future Work}

\subsection{Remaining Tasks}

\begin{enumerate}
    \item Develop Simulink models for all three rectifier configurations
    \item Implement R, RL, and highly inductive loads
    \item Design and simulate firing circuits
    \item Analyze current continuity conditions
    \item Evaluate free-wheeling diode performance
    \item Study four-quadrant operation and regenerative braking
    \item Generate comparative plots and performance tables
    \item Validate against analytical calculations
\end{enumerate}


% ==================== CONCLUSION ====================
\section{Conclusion}

\subsection{Summary of Work}
\TODO{Summarize what was accomplished in both parts:}

This project investigated controlled rectifier circuits for two main applications:

\paragraph{Part I - Battery Charger Design:}
\begin{itemize}
    \item Developed MATLAB functions for three rectifier configurations
    \item Analyzed the relationship between firing angle and charging time
    \item Compared half-wave, center-tapped, and bridge rectifiers
    \item Investigated optional features: SoC tracking and loss analysis
\end{itemize}

\paragraph{Part II - Load Analysis:}
\begin{itemize}
    \item Simulated controlled rectifiers with R, RL, and highly inductive loads
    \item Studied current continuity in different operating modes
    \item Analyzed the role of free-wheeling diodes
    \item Investigated four-quadrant operation and regenerative modes
    \item Designed and implemented firing circuits
\end{itemize}

\subsection{Key Findings}

\subsubsection{Part I Findings}
\TODO{List major conclusions from Part I:}
\begin{enumerate}
    \item \textbf{Firing Angle Effect:}
    \begin{itemize}
        \item Increasing firing angle decreases average output voltage
        \item Results in longer charging time
        \item Provides control over charging rate
    \end{itemize}
    
    \item \textbf{Configuration Comparison:}
    \begin{itemize}
        \item Full-wave rectifiers provide twice the average voltage of half-wave
        \item Charging time is approximately halved with full-wave
        \item Bridge and center-tapped perform similarly electrically
        \item Choice depends on transformer availability and cost
    \end{itemize}
    
    \item \textbf{Practical Considerations:}
    \begin{itemize}
        \item Thyristor losses are \TODO{significant/negligible}
        \item SoC tracking enables better charge control
        \item \TODO{Add other findings}
    \end{itemize}
\end{enumerate}

\subsubsection{Part II Findings}
\TODO{List major conclusions from Part II:}
\begin{enumerate}
    \item \textbf{Load Type Impact:}
    \begin{itemize}
        \item Resistive loads: Simple, discontinuous current, voltage and current in phase
        \item RL loads: Complex behavior, possible continuous conduction, current lag
        \item Highly inductive: Constant current, always CCM, enables regeneration
    \end{itemize}
    
    \item \textbf{Current Continuity:}
    \begin{itemize}
        \item Determined by L/R ratio, firing angle, and rectifier type
        \item Full-wave more likely to have CCM than half-wave
        \item Important for predicting output characteristics
    \end{itemize}
    
    \item \textbf{Free-Wheeling Diode:}
    \begin{itemize}
        \item Improves output voltage waveform
        \item Eliminates negative voltage excursions
        \item Prevents regenerative operation
        \item Essential for motor drives operating only in Quadrant I
    \end{itemize}
    
    \item \textbf{Four-Quadrant Operation:}
    \begin{itemize}
        \item $\alpha > 90°$ enables regenerative mode
        \item Negative voltage with positive current observed
        \item Important for braking in motor drives
        \item Not possible with free-wheeling diode
    \end{itemize}
    
    \item \textbf{Configuration Comparison:}
    \begin{itemize}
        \item Full-wave superior to half-wave in most aspects
        \item Center-tapped vs. bridge: trade-off between component count and transformer requirements
        \item \TODO{Add specific performance differences observed}
    \end{itemize}
\end{enumerate}

\subsection{Achievement of Objectives}

\TODO{Evaluate whether objectives were met:}

\begin{table}[H]
\centering
\caption{Objective Achievement Matrix}
\label{tab:objectives_achievement}
\begin{tabular}{@{}p{8cm}p{2cm}p{3cm}@{}}
\toprule

\end{tabular}
\end{table}

\subsection{Practical Implications}

\subsubsection{Design Guidelines}
\TODO{Provide practical design recommendations:}

\paragraph{For Battery Charging Applications:}
\begin{itemize}
    \item Use full-wave bridge rectifier for best performance
    \item Operate at moderate firing angles (\TODO{e.g., 30-60°}) for balance of speed and control
    \item Implement SoC monitoring for optimal charging
    \item Consider thyristor losses in high-power applications
\end{itemize}

\paragraph{For Motor Drive Applications:}
\begin{itemize}
    \item Use full-wave configuration for better performance
    \item Highly inductive armature enables good control
    \item Include free-wheeling diode for unidirectional operation
    \item Omit FWD if regenerative braking required
    \item Design for continuous conduction mode
\end{itemize}

\paragraph{For General DC Power Supplies:}
\begin{itemize}
    \item Match rectifier type to load characteristics
    \item Consider harmonic filtering requirements
    \item Balance cost vs. performance
    \item Provide adequate voltage rating for thyristors
\end{itemize}

\subsubsection{Economic Considerations}
\TODO{Discuss cost-benefit aspects:}
\begin{itemize}
    \item Component count vs. transformer requirements
    \item Efficiency impact on operating costs
    \item Reliability and maintenance
    \item Application-specific optimization
\end{itemize}

\subsection{Lessons Learned}

\TODO{Reflect on technical insights gained:}

\subsubsection{Technical Insights}
\begin{enumerate}
    \item \TODO{e.g., Importance of load type in rectifier design}
    \item \TODO{Trade-offs between control range and efficiency}
    \item \TODO{Complexity of inductive load behavior}
    \item \TODO{Value of simulation for understanding transient behavior}
\end{enumerate}

\subsubsection{Analytical Skills}
\begin{enumerate}
    \item \TODO{Developed proficiency in MATLAB function development}
    \item \TODO{Gained experience with Simulink power systems modeling}
    \item \TODO{Enhanced understanding of analytical vs. simulation approaches}
    \item \TODO{Improved ability to validate results}
\end{enumerate}

\subsection{Future Work}

\subsubsection{Short-Term Extensions}
\TODO{Suggest immediate improvements:}
\begin{itemize}
    \item Implement closed-loop control
    \item Add more sophisticated battery models
    \item Include harmonic analysis
    \item Develop efficiency optimization algorithms
    \item Create GUI for parameter adjustment
\end{itemize}

\subsubsection{Long-Term Research Directions}
\TODO{Propose advanced topics:}
\begin{itemize}
    \item Three-phase controlled rectifiers
    \item PWM rectifiers for improved power quality
    \item Active power factor correction
    \item Soft-switching techniques
    \item Digital control implementation
    \item Hardware prototype development
    \item Experimental validation
\end{itemize}

\subsubsection{Related Topics}
\TODO{Connect to broader field:}
\begin{itemize}
    \item DC-DC converters for post-regulation
    \item Inverters for AC motor drives
    \item Modern power electronic converters
    \item Renewable energy systems
    \item Electric vehicle charging infrastructure
\end{itemize}

\subsection{Relevance to Modern Power Electronics}

\TODO{Discuss how this project relates to current technology:}

\paragraph{Continued Relevance:}
\begin{itemize}
    \item Thyristor rectifiers still used in high-power applications
    \item Fundamental concepts apply to modern converters
    \item Understanding of controlled rectification essential
    \item Basis for more advanced topologies
\end{itemize}

\paragraph{Evolution to Modern Solutions:}
\begin{itemize}
    \item IGBTs and MOSFETs for faster switching
    \item PWM techniques for better waveforms
    \item Digital control for enhanced performance
    \item Integration with renewable energy
\end{itemize}

\subsection{Final Remarks}

\TODO{Conclude with overall assessment:}

This project provided comprehensive understanding of controlled rectifier operation through both analytical (MATLAB) and simulation (Simulink) approaches. The dual methodology of Part I (analytical functions) and Part II (detailed simulation) offered complementary insights into rectifier behavior.

Key achievements:
\begin{itemize}
    \item Successful implementation of battery charger design functions
    \item Detailed analysis of load type effects on rectifier performance
    \item Validation of theoretical concepts through simulation
    \item Development of practical design guidelines
\end{itemize}

The knowledge gained forms a solid foundation for understanding power electronic converters and their applications in modern electrical systems.

\TODO{Add personal reflection on the project experience and learning outcomes}


\newpage

% ==================== APPENDIX ====================
\appendix

\section{MATLAB Code}

This appendix contains the complete MATLAB implementation for the battery charger analysis.

\subsection{System Parameters (params.m)}

\begin{lstlisting}
% default battery_charger params

% Supply
Vrms = 230; 
f = 50;

% Battery
Vbat = 20; 
Rbat = 0.1; 
capacity = 50; 
capUnit = 'Ah';

% SoC & time
SoC_init = 12; 
SoC_target = 100;
t_charging_hours =  0.5; % Simulated User Input
t_charge = inf;

% Thyristor
alpha = 30;
alpha_deg = 0:2:180;         
Vt = 1.5;       
Rth = 0.001;  
Ileak = 0.01;  
t_rise = 1e-6;  
t_fall = 2e-6;  

% Simulation params 
dt = 1/(300*f);

% Visualization
enablePlots = true;
savePlots = true; 
\end{lstlisting}

\subsection{Half-Wave Rectifier (half\_wave\_charger.m)}

\begin{lstlisting}[basicstyle=\ttfamily\tiny]
function [alpha_deg, charging_time_hours, SoC_final, P_loss_avg, metrics] = half_wave_charger(Vrms, f, Vbat, Rbat, capacity, capUnit, varargin)
% Syntax:
%   [alpha_deg, charging_time_hours] = half_wave_charger(Vrms, f, Vbat, Rbat, capacity, capUnit)
%   [alpha_deg, charging_time_hours, SoC_final, P_loss_avg, metrics] = ...
%       half_wave_charger(..., 't_charge', t_charge, 'SoC_init', SoC_init, 'Vt', Vt, 'Rth', Rth)
%
% Inputs:
%   Vrms        - Supply voltage RMS value [V]
%   f           - Supply frequency [Hz]
%   Vbat        - Battery nominal voltage [V]
%   Rbat        - Battery internal resistance [Ohm]
%   capacity    - Battery capacity value
%   capUnit     - Battery capacity unit 'Ah' or 'Wh'
%
% Name-Value Optional Inputs:
%   't_charge'  - Charging time [Hours].
%   'SoC_init'  - Initial state of charge [%]
%   'Vt'        - Thyristor forward drop [V]
%   'Rth'       - Equivalent on-state resistance of thyristor [Ohm]
%   'Ileak'     - Reverse leakage Current [A]
%   't_rise'    - Rise time [s]
%   't_fall'    - Fall time[s]
%   'alpha_deg' - Vector of firing angles to analyze [deg] (for sweep analysis)
%   'alpha'     - Single firing angle [deg] (for detailed waveform plots)
%   'SoC_target'- Target state of charge [%]
%   'enablePlots'- Enable/disable all plots [boolean]
%
% Outputs:
%   alpha_deg           - Analyzed firing angles [deg] (vector)
%   charging_time_hours - Charging time for each alpha [h] (vector, same size as alpha_deg)
%   SoC_final           - Final SoC if 't_charge' provided [%] (vector)
%   P_loss_avg          - Average SCR conduction loss for each alpha [W] (vector)
%   metrics             - Struct with fields per alpha: Vavg, Vrms, Iavg, Irms,
%                         P_batt, P_thyristor, P_blocking, P_switching, P_total

% ---------- Load parameters from workspace ----------
try
    SoC_target_ws = evalin('base', 'SoC_target');
catch
    SoC_target_ws = 80; 
end
try
    t_charge_ws = evalin('base', 't_charge');
    if isinf(t_charge_ws)
        t_charge_ws = [];
    end
catch
    t_charge_ws = []; 
end
try
    alpha_deg_ws = evalin('base', 'alpha_deg');
catch
    alpha_deg_ws = 0:5:175;
end
try
    enablePlots_ws = evalin('base', 'enablePlots');
catch
    enablePlots_ws = false; 
end
try
    savePlots_ws = evalin('base', 'savePlots');
catch
    savePlots_ws = false;
end
try
    Rth_ws = evalin('base', 'Rth');
catch
    Rth_ws = 0;
end
try
    alpha_ws = evalin('base', 'alpha');
catch
    alpha_ws = 90;
end
try
    SoC_init_ws = evalin('base', 'SoC_init');
catch
    SoC_init_ws = 20;
end

% ---------- Parse inputs ----------
p = inputParser;
addParameter(p, 't_charge', t_charge_ws, @(x) isempty(x) || (isnumeric(x) && isscalar(x)));
addParameter(p, 'SoC_init', SoC_init_ws, @isnumeric);
addParameter(p, 'SoC_target', SoC_target_ws, @isnumeric);
addParameter(p, 'Vt', 0, @isnumeric);
addParameter(p, 'alpha_deg', alpha_deg_ws, @(x) isnumeric(x) && isvector(x));
addParameter(p, 'Rth', Rth_ws, @isnumeric);
addParameter(p, 'Ileak', 0, @isnumeric); 
addParameter(p, 't_rise', 0, @isnumeric); 
addParameter(p, 't_fall', 0, @isnumeric);
addParameter(p, 'alpha', alpha_ws, @isnumeric); 
addParameter(p, 'enablePlots', enablePlots_ws, @(x) islogical(x) || isnumeric(x));
addParameter(p, 'savePlots', savePlots_ws, @(x) islogical(x) || isnumeric(x));
parse(p, varargin{:});

t_charge      = p.Results.t_charge;
SoC_init      = p.Results.SoC_init;
SoC_target    = p.Results.SoC_target;
Vt            = p.Results.Vt;
alpha_deg     = p.Results.alpha_deg;
Rth           = p.Results.Rth;
Ileak         = p.Results.Ileak;
t_rise        = p.Results.t_rise;
t_fall        = p.Results.t_fall;
enablePlots   = logical(p.Results.enablePlots);
savePlots     = logical(p.Results.savePlots);
alpha         = p.Results.alpha;

if isstring(capUnit) || ischar(capUnit)
    if strcmpi(string(capUnit), "Wh")
        capacity = capacity / Vbat; % Wh -> Ah
    end
end

t_charge = t_charge*3600; % convert to Seconds

% ---------- Constants ----------
Vm = sqrt(2) * Vrms; % Peak voltage

% ---------- Sampling ----------
N = 4096;
theta = linspace(0, 2*pi, N+1); theta(end) = []; % exclude endpoint

% Half-wave: source voltage (positive half only, zero for negative)
V_source = Vm * sin(theta);
V_source(V_source < 0) = 0;  % Rectification

Q_C  = capacity * 3600;     % Ah -> Coulombs

% ---------- Outputs ----------
na = numel(alpha_deg);
Vavg = zeros(1, na);
Vout_rms = zeros(1, na);
Iavg = zeros(1, na);
Irms = zeros(1, na);
P_loss_avg = zeros(1, na);
P_batt = zeros(1, na);        % Battery internal losses
P_thyristor = zeros(1, na);   % Thyristor conduction losses
P_blocking = zeros(1, na);    % Thyristor blocking/leakage losses
P_switching = zeros(1, na);   % Thyristor switching losses
P_total = zeros(1, na);       % Total power losses
charging_time_hours = zeros(1, na);
SoC_final = zeros(1, na);

for k = 1:na
    a = deg2rad(alpha_deg(k));

    % Gate available only in positive half-cycle after firing angle
    gate = (theta >= a) & (theta <= pi);

    v_conv = V_source - Vt;
    v_conv(v_conv < 0) = 0;

    % Conduction only when above battery clamp
    cond  = v_conv > Vbat;

    i_t = zeros(size(theta));
    on  = gate & cond;
    i_t(on) = (v_conv(on) - Vbat) ./ Rbat;

    v_out = zeros(size(theta));
    v_out(on) = v_conv(on);

    Vavg(k) = mean(v_out);
    Vout_rms(k) = sqrt(mean(v_out.^2));
    Iavg(k) = mean(i_t);
    Irms(k) = sqrt(mean(i_t.^2));

    % Power Loss Calculations
    P_batt(k) = Irms(k)^2 * Rbat;
    
    if ~(Ileak == 0)
        P_thyristor(k) = Vt*Iavg(k) + Rth*Irms(k)^2;
    end
    
    if Ileak > 0
        v_blocking = V_source;
        v_blocking(on) = 0;  % Zero when conducting
        P_blocking(k) = mean(v_blocking) * Ileak;
    else
        P_blocking(k) = 0;
    end
    
    if (t_rise > 0 || t_fall > 0) && Irms(k) > 0
        I_peak = max(i_t);
        V_block_avg = mean(V_source(~on));
        if isnan(V_block_avg)
            V_block_avg = Vm;
        end
        E_on = (1/6) * V_block_avg * I_peak * t_rise;
        E_off = (1/6) * V_block_avg * I_peak * t_fall;
        P_switching(k) = f * (E_on + E_off);
    else
        P_switching(k) = 0;
    end
    
    P_total(k) = P_batt(k) + P_thyristor(k) + P_blocking(k) + P_switching(k);
    P_loss_avg(k) = P_thyristor(k);

    if isempty(t_charge) || isinf(t_charge)
        dSoC = max(SoC_target - SoC_init, 0)/100; 
        if Iavg(k) > 0
            t_sec = (Q_C * dSoC) / Iavg(k);
        else
            t_sec = inf;
        end
        charging_time_hours(k) = t_sec/3600;
        SoC_final(k) = SoC_target;
    else
        t_sec = t_charge;
        SoC_final(k) = min(100, SoC_init + 100*(Iavg(k)*t_sec)/Q_C);
        charging_time_hours(k) = t_sec/3600;
    end
end

metrics = struct('Vavg', Vavg, 'Vrms', Vout_rms, 'Iavg', Iavg, 'Irms', Irms, ...
                 'P_batt', P_batt, 'P_thyristor', P_thyristor, ...
                 'P_blocking', P_blocking, 'P_switching', P_switching, 'P_total', P_total);

[~, alpha_idx] = min(abs(alpha_deg - alpha));

% Create output directory if savePlots is enabled
if savePlots
    output_dir = fullfile(fileparts(mfilename('fullpath')), '..', 'figures', 'half_wave');
    if ~exist(output_dir, 'dir')
        mkdir(output_dir);
    end
end

% Plotting and display code follows...
% (Abbreviated for space - full implementation includes comprehensive plotting)

end
\end{lstlisting}

\subsection{Full-Wave Center-Tapped (full\_wave\_ct\_charger.m)}

\begin{lstlisting}[basicstyle=\ttfamily\tiny]
function [alpha_deg, charging_time_hours, SoC_final, P_loss_avg, metrics] = full_wave_ct_charger(Vrms, f, Vbat, Rbat, capacity, capUnit, varargin)
% Full-wave center-tapped rectifier - uses two thyristors
% Key difference from half-wave: V_abs = Vm * abs(sin(theta))
% Output frequency doubled (100 Hz ripple vs 50 Hz)
% Average voltage: Vdc = (2*Vm/pi) * (1 + cos(alpha)) - Vt

% Load workspace parameters
try SoC_target_ws = evalin('base', 'SoC_target'); catch, SoC_target_ws = 80; end
try t_charge_ws = evalin('base', 't_charge'); if isinf(t_charge_ws), t_charge_ws = []; end
catch, t_charge_ws = []; end
try alpha_deg_ws = evalin('base', 'alpha_deg'); catch, alpha_deg_ws = 0:5:175; end
try enablePlots_ws = evalin('base', 'enablePlots'); catch, enablePlots_ws = false; end
try savePlots_ws = evalin('base', 'savePlots'); catch, savePlots_ws = false; end
try Rth_ws = evalin('base', 'Rth'); catch, Rth_ws = 0; end
try alpha_ws = evalin('base', 'alpha'); catch, alpha_ws = 90; end
try SoC_init_ws = evalin('base', 'SoC_init'); catch, SoC_init_ws = 90; end

% Parse inputs
p = inputParser;
addParameter(p, 't_charge', t_charge_ws, @(x) isempty(x) || (isnumeric(x) && isscalar(x)));
addParameter(p, 'SoC_init', SoC_init_ws, @isnumeric);
addParameter(p, 'SoC_target', SoC_target_ws, @isnumeric);
addParameter(p, 'Vt', 0, @isnumeric);
addParameter(p, 'alpha_deg', alpha_deg_ws, @(x) isnumeric(x) && isvector(x));
addParameter(p, 'Rth', Rth_ws, @isnumeric);
addParameter(p, 'Ileak', 0, @isnumeric); 
addParameter(p, 't_rise', 0, @isnumeric); 
addParameter(p, 't_fall', 0, @isnumeric);
addParameter(p, 'alpha', alpha_ws, @isnumeric); 
addParameter(p, 'enablePlots', enablePlots_ws, @(x) islogical(x) || isnumeric(x));
addParameter(p, 'savePlots', savePlots_ws, @(x) islogical(x) || isnumeric(x));
parse(p, varargin{:});

t_charge = p.Results.t_charge; SoC_init = p.Results.SoC_init;
SoC_target = p.Results.SoC_target; Vt = p.Results.Vt;
alpha_deg = p.Results.alpha_deg; Rth = p.Results.Rth;
Ileak = p.Results.Ileak; t_rise = p.Results.t_rise; t_fall = p.Results.t_fall;
enablePlots = logical(p.Results.enablePlots); savePlots = logical(p.Results.savePlots);
alpha = p.Results.alpha;

if isstring(capUnit) || ischar(capUnit)
    if strcmpi(string(capUnit), "Wh"), capacity = capacity / Vbat; end
end
t_charge = t_charge*3600; % Hours to seconds

% Constants and sampling
Vm = sqrt(2) * Vrms;
N = 4096;
theta = linspace(0, 2*pi, N+1); theta(end) = [];
theta_mod = mod(theta, pi);
V_abs = Vm * abs(sin(theta)); % Full-wave rectification
Q_C = capacity * 3600;

% Initialize outputs
na = numel(alpha_deg);
Vavg = zeros(1, na); Vout_rms = zeros(1, na);
Iavg = zeros(1, na); Irms = zeros(1, na);
P_loss_avg = zeros(1, na); P_batt = zeros(1, na);
P_thyristor = zeros(1, na); P_blocking = zeros(1, na);
P_switching = zeros(1, na); P_total = zeros(1, na);
charging_time_hours = zeros(1, na); SoC_final = zeros(1, na);

for k = 1:na
    a = deg2rad(alpha_deg(k));
    gate = theta_mod >= a & theta_mod <= pi;
    v_conv = (V_abs - Vt);
    cond = v_conv > Vbat;
    
    i_t = zeros(size(theta));
    on = gate & cond;
    i_t(on) = (v_conv(on) - Vbat) ./ Rbat;
    
    v_out = zeros(size(theta));
    v_out(on) = v_conv(on);
    
    Vavg(k) = mean(v_out); Vout_rms(k) = sqrt(mean(v_out.^2));
    Iavg(k) = mean(i_t); Irms(k) = sqrt(mean(i_t.^2));
    
    % Power losses
    P_batt(k) = Irms(k)^2 * Rbat;
    if ~(Ileak == 0), P_thyristor(k) = Vt*Iavg(k) + Rth*Irms(k)^2; end
    if Ileak > 0
        v_blocking = V_abs; v_blocking(on) = 0;
        P_blocking(k) = mean(v_blocking) * Ileak;
    else, P_blocking(k) = 0; end
    
    if (t_rise > 0 || t_fall > 0) && Irms(k) > 0
        I_peak = max(i_t); V_block_avg = mean(V_abs(~on));
        if isnan(V_block_avg), V_block_avg = Vm; end
        E_on = (1/6) * V_block_avg * I_peak * t_rise;
        E_off = (1/6) * V_block_avg * I_peak * t_fall;
        P_switching(k) = f * (E_on + E_off);
    else, P_switching(k) = 0; end
    
    P_total(k) = P_batt(k) + P_thyristor(k) + P_blocking(k) + P_switching(k);
    P_loss_avg(k) = P_thyristor(k);

    % Calculate charging time or final SoC
    if isempty(t_charge) || isinf(t_charge)
        dSoC = max(SoC_target - SoC_init, 0)/100; 
        if Iavg(k) > 0, t_sec = (Q_C * dSoC) / Iavg(k);
        else, t_sec = inf; end
        charging_time_hours(k) = t_sec/3600;
        SoC_final(k) = SoC_target;
    else
        t_sec = t_charge;
        SoC_final(k) = min(100, SoC_init + 100*(Iavg(k)*t_sec)/Q_C);
        charging_time_hours(k) = t_sec/3600;
    end
end

metrics = struct('Vavg', Vavg, 'Vrms', Vout_rms, 'Iavg', Iavg, 'Irms', Irms, ...
                 'P_batt', P_batt, 'P_thyristor', P_thyristor, ...
                 'P_blocking', P_blocking, 'P_switching', P_switching, 'P_total', P_total);

[~, alpha_idx] = min(abs(alpha_deg - alpha));

% Create output directory
if savePlots
    output_dir = fullfile(fileparts(mfilename('fullpath')), '..', 'figures', 'full_wave_ct');
    if ~exist(output_dir, 'dir'), mkdir(output_dir); end
end

% Plotting code similar to half_wave_charger follows...
% (Abbreviated for space - includes voltage/current/loss plots)

end
\end{lstlisting}

\subsection{Full-Wave Bridge (full\_wave\_bridge\_charger.m)}

\begin{lstlisting}[basicstyle=\ttfamily\tiny]
function [alpha_deg, charging_time_hours, SoC_vec, P_loss_vec] = full_wave_bridge_charger(Vrms, f, Vbat, Rbat, capacity, capUnit, varargin)
% FULL_WAVE_BRIDGE_CHARGER Analyzes full-wave bridge controlled rectifier
% Bridge configuration with 4 thyristors
% Two thyristor drops in series: Vdc = (2*Vm/pi)*(1+cos(alpha)) - 2*Vt

p = inputParser;
addParameter(p, 't_charge', inf, @isnumeric);
addParameter(p, 'SoC_init', 20, @isnumeric);
addParameter(p, 'Vt', 0, @isnumeric);
addParameter(p, 'Ileak', 0, @isnumeric);
addParameter(p, 't_rise', 0, @isnumeric);
addParameter(p, 't_fall', 0, @isnumeric);
addParameter(p, 'alpha_given', [], @isnumeric);
parse(p, varargin{:});

t_charge = p.Results.t_charge;
SoC_init = p.Results.SoC_init;
Vt = p.Results.Vt;
Ileak = p.Results.Ileak;

Vm = sqrt(2) * Vrms;
omega = 2 * pi * f;

alpha_deg = 0:5:150;
alpha_rad = deg2rad(alpha_deg);

switch lower(string(capUnit))
    case "ah"
        Q_tot = capacity * 3600;
    case "wh"
        E_tot = capacity * 3600;
        Q_tot = E_tot / Vbat;
    otherwise
        error('capUnit must be ''Ah'' or ''Wh''.');
end

SoC_target = 80;
Q_init = (SoC_init/100) * Q_tot;

nAlpha = numel(alpha_deg);
charging_time_hours = nan(1, nAlpha);
SoC_vec = nan(1, nAlpha);
P_loss_vec = nan(1, nAlpha);

nonideal_SCR = ~all(ismember({'Vt','Ileak','t_rise','t_fall'}, p.UsingDefaults));

for k = 1:nAlpha
    alpha = alpha_rad(k);
    Vdc_ideal = (Vm/pi) * (1+cos(alpha));
    Vdc_eff = Vdc_ideal - 2*Vt;  % Two thyristor drops
    I_charge = (Vdc_eff - Vbat) / Rbat;
    
    if I_charge <= 0 || Vdc_eff <= Vbat
        charging_time_hours(k) = Inf;
        if nonideal_SCR
            P_loss_vec(k) = Vrms * Ileak;
        else
            P_loss_vec(k) = 0;
        end
        continue;
    end

    if isinf(t_charge)
        dSoC = SoC_target - SoC_init;
        if dSoC <= 0
            t_req = 0;
            SoC_vec(k) = SoC_init;
        else
            dQ = (dSoC/100) * Q_tot;
            t_req = dQ / I_charge;
            SoC_vec(k) = SoC_target;
        end
        charging_time_hours(k) = t_req / 3600;
    else
        dQ = I_charge * t_charge;
        SoC_f = SoC_init + 100 * (dQ / Q_tot);
        SoC_f = min(max(SoC_f, 0), 100);
        SoC_vec(k) = SoC_f;
        
        dQ_req = (SoC_target - SoC_init)/100 * Q_tot;
        if dQ_req <= 0
            t_req = 0;
        else
            t_req = dQ_req / I_charge;
        end
        charging_time_hours(k) = t_req / 3600;
    end

    if nonideal_SCR
        P_cond = 2 * Vt * I_charge;
        P_leak = Vrms * Ileak;
        P_loss_vec(k) = P_cond + P_leak;
    else
        P_loss_vec(k) = 0;
    end
end

% Generate output plots
figure('Name', 'Full-Wave Bridge Rectifier - Firing Angle vs Charging Time');
plot(alpha_deg, charging_time_hours, 'g-d', 'LineWidth', 2);
grid on;
xlabel('Firing Angle \alpha (degrees)');
ylabel('Charging Time (hours)');
title('Full-Wave Bridge Controlled Rectifier for Battery Charging');
legend('Charging Time');

end
\end{lstlisting}

\newpage
\section{Load Analysis MATLAB Code}

This section contains the MATLAB implementation for the load analysis (Milestone 2) including parameter sweeps and automated Simulink simulation control.

\subsection{Load Analysis Parameters (params.m)}

\begin{lstlisting}
% Load Analysis Parameters

% Supply
Vrms = 230; 
f = 50;
alphas_deg = 0:5:180;            
simTime = 0.1;
T = 1/f;

% Load scenarios
scenarios = struct();
scenarios(1).name = 'resistive_only';
scenarios(1).R = 10;           % Resistance only (ohms)
scenarios(1).L = 1e-6;         % Minimal inductance (H)

scenarios(2).name = 'R_L_load';
scenarios(2).R = 10;           % Resistance (ohms)
scenarios(2).L = 50e-3;        % Moderate inductance (H)

scenarios(3).name = 'highly_inductive';
scenarios(3).R = 5;            % Lower resistance (ohms)
scenarios(3).L = 200e-3;       % High inductance (H)

% Pulse Generator parameters
pulse_amplitude = 10;           % Gate signal amplitude (V)
pulse_width = 50;              % Pulse width (% of period)
pulse_period = 1/f;            % Period based on line frequency (s)

% Model selection: 'all', 'ct', 'bg', 'hf', or cell array {'ct', 'bg'}
% 'ct' = center_taped, 'bg' = bridge (Full_Wave_Bridge), 'hf' = half_wave
modelSelection = 'ct';        

% Live plotting option
enableLivePlot = false;  

% Graph generation option
generateGraphs = true;   
graphAlphas = [30, 90, 180];  % Alpha values to plot (degrees)

R = 30;
L = 0.1;

phase_delay = 0.005;
phase_delay2 = 0.015;
\end{lstlisting}

\subsection{Load Analysis Sweep Script (load\_analysis\_sweep.m)}

\begin{lstlisting}[basicstyle=\ttfamily\tiny]
run params.m

% Close all open Simulink models
bd = find_system('type', 'block_diagram');
for i = 1:length(bd)
    if ~strcmp(bd{i}, 'simulink')
        try
            close_system(bd{i}, 0);
        catch
        end
    end
end

% Setup directories
saveFolder = fullfile(fileparts(mfilename('fullpath')),'results');
if ~exist(saveFolder,'dir')
    mkdir(saveFolder);
end

scriptDir = fileparts(mfilename('fullpath'));
slxDir = fullfile(scriptDir,'..','simulink');
files = dir(fullfile(slxDir,'*.slx'));
if isempty(files)
    error('No .slx models found in %s', slxDir);
end

models = {files.name};

% Filter models based on selection
if ~strcmp(modelSelection, 'all')
    modelMap = struct('ct', 'center_taped', 'bg', 'Full_Wave_Bridge', 'hf', 'half_wave');
    
    if ischar(modelSelection)
        if isfield(modelMap, modelSelection)
            targetName = modelMap.(modelSelection);
            models = models(contains(lower(models), lower(targetName)));
        else
            error('Invalid modelSelection: %s', modelSelection);
        end
    elseif iscell(modelSelection)
        selectedModels = {};
        for i = 1:length(modelSelection)
            sel = modelSelection{i};
            if isfield(modelMap, sel)
                targetName = modelMap.(sel);
                selectedModels = [selectedModels; models(contains(lower(models), lower(targetName)))];
            end
        end
        models = unique(selectedModels);
    end
end

fprintf('Found %d model(s):\n', numel(models));
for k=1:numel(models)
    fprintf(' - %s\n', models{k});
end

fprintf('\nLoad scenarios:\n');
for s=1:numel(scenarios)
    fprintf(' %d. %s: R=%.2f Ohm, L=%.1f mH\n', s, scenarios(s).name, scenarios(s).R, scenarios(s).L*1000);
end

if enableLivePlot
    figHandle = figure('Name', 'Live Waveforms', 'NumberTitle', 'off');
end

results = struct();

% Main sweep loop
for si = 1:numel(scenarios)
    scenario = scenarios(si);
    fprintf('\n========== Scenario %d: %s ==========\n', si, scenario.name);
    
    for mi = 1:numel(models)
        modelName = models{mi};
        modelPath = fullfile(slxDir, modelName);
        [~,modelBase] = fileparts(modelName);
        
        fprintf('\nRunning model: %s\n', modelName);
        
        % Close existing instance if loaded
        if bdIsLoaded(modelBase)
            fprintf('  Closing existing model instance...\n');
            try
                simStatus = get_param(modelBase, 'SimulationStatus');
                if ~strcmp(simStatus, 'stopped')
                    set_param(modelBase, 'SimulationCommand', 'stop');
                    pause(1);
                end
            catch ME
                warning('Error stopping simulation: %s', ME.message);
            end
            
            try
                close_system(modelBase, 0);
            catch ME
                warning('Could not close model: %s', ME.message);
            end
        end
        
        load_system(modelPath);
        modelResults = cell(numel(alphas_deg),1);
        
        % Alpha sweep
        for ai = 1:numel(alphas_deg)
            alpha_deg = alphas_deg(ai);
            phase_delay = (alpha_deg/360) * (1/f);
            phase_delay2 = phase_delay + (1/(2*f));
            
            % Assign parameters to base workspace
            assignin('base','alpha_deg',alpha_deg);
            assignin('base','phase_delay',phase_delay);
            assignin('base','f',f);
            assignin('base','Vrms',Vrms);
            assignin('base','R',scenario.R);
            assignin('base','L',scenario.L);
            assignin('base','pulse_amplitude',pulse_amplitude);
            assignin('base','pulse_width',pulse_width);
            assignin('base','pulse_period',pulse_period);
            
            evalin('base', 'clear Vout Iout V_data I_data vout iout');
            
            fprintf('  alpha = %3d deg ... ', alpha_deg);
            
            try
                simOut = sim(modelBase, 'StopTime', num2str(simTime), ...
                    'SaveOutput','on', 'SaveTime','on', 'ReturnWorkspaceOutputs','on');
            catch ME
                fprintf('FAILED\n');
                warning('Simulation failed: %s', ME.message);
                modelResults{ai} = struct('alpha_deg',alpha_deg,'error',ME);
                continue
            end
            
            fprintf('OK\n');
            
            % Store results
            entry.alpha_deg = alpha_deg;
            entry.phase_delay = phase_delay;
            entry.R = scenario.R;
            entry.L = scenario.L;
            entry.simOut = simOut;
            
            % Extract voltage and current data
            try
                V_data = []; I_data = []; t_data = simOut.tout;
                
                % Try to get data from simulation output
                if isprop(simOut, 'Vout') && isprop(simOut, 'Iout')
                    Vout = simOut.Vout;
                    Iout = simOut.Iout;
                    
                    if isa(Vout, 'timeseries')
                        V_data = Vout.Data;
                        t_data = Vout.Time;
                    elseif isnumeric(Vout)
                        V_data = Vout(:);
                    end
                    
                    if isa(Iout, 'timeseries')
                        I_data = Iout.Data;
                    elseif isnumeric(Iout)
                        I_data = Iout(:);
                    end
                end
                
                % Calculate metrics
                if ~isempty(V_data) && ~isempty(I_data)
                    entry.Vavg = mean(V_data);
                    entry.Vrms = sqrt(mean(V_data.^2));
                    entry.Iavg = mean(I_data);
                    entry.Irms = sqrt(mean(I_data.^2));
                    
                    % Live plotting
                    if enableLivePlot && mod(ai, 6) == 1
                        figure(figHandle);
                        subplot(2,1,1);
                        plot(t_data, V_data, 'b-', 'LineWidth', 1.5);
                        xlabel('Time (s)'); ylabel('Voltage (V)');
                        title(sprintf('%s - %s: alpha=%d deg', modelBase, scenario.name, alpha_deg));
                        grid on;
                        
                        subplot(2,1,2);
                        plot(t_data, I_data, 'r-', 'LineWidth', 1.5);
                        xlabel('Time (s)'); ylabel('Current (A)');
                        title(sprintf('Current at alpha=%d deg', alpha_deg));
                        grid on;
                        drawnow;
                    end
                else
                    entry.Vavg = NaN; entry.Vrms = NaN;
                    entry.Iavg = NaN; entry.Irms = NaN;
                end
            catch ME
                warning('Error extracting data: %s', ME.message);
                entry.Vavg = NaN; entry.Vrms = NaN;
                entry.Iavg = NaN; entry.Irms = NaN;
            end
            
            modelResults{ai} = entry;
        end
        
        % Save results
        fieldName = sprintf('%s_%s', modelBase, scenario.name);
        results.(fieldName) = modelResults;
        saveFile = fullfile(saveFolder, sprintf('%s_%s_results.mat', modelBase, scenario.name));
        save(saveFile, 'modelResults','scenario','-v7.3');
        fprintf('Saved results to %s\n', saveFile);
        
        % Print summary
        fprintf('\nSummary for %s - %s:\n', modelName, scenario.name);
        fprintf('  Alpha(deg)  Vavg(V)   Vrms(V)   Iavg(A)   Irms(A)\n');
        fprintf('  -----------------------------------------------\n');
        for ai = 1:min(5, numel(modelResults))
            if ~isempty(modelResults{ai}) && isfield(modelResults{ai}, 'Vavg')
                e = modelResults{ai};
                fprintf('  %6d     %7.2f   %7.2f   %7.3f   %7.3f\n', ...
                    e.alpha_deg, e.Vavg, e.Vrms, e.Iavg, e.Irms);
            end
        end
        
        % Generate graphs for selected alpha values
        if generateGraphs
            fprintf('\nGenerating graphs...\n');
            figFolder = fullfile(saveFolder, '..', '..', 'figures', modelBase, scenario.name);
            if ~exist(figFolder, 'dir')
                mkdir(figFolder);
            end
            
            for alpha_val = graphAlphas
                resultIdx = -1;
                for ai = 1:numel(modelResults)
                    if ~isempty(modelResults{ai}) && modelResults{ai}.alpha_deg == alpha_val
                        resultIdx = ai;
                        break;
                    end
                end
                
                if resultIdx > 0 && isfield(modelResults{resultIdx}, 'simOut')
                    result = modelResults{resultIdx};
                    simOut = result.simOut;
                    
                    % Extract plot data
                    if isprop(simOut, 'Vout') && isprop(simOut, 'Iout')
                        VoutData = simOut.Vout;
                        IoutData = simOut.Iout;
                        
                        if isa(VoutData, 'timeseries')
                            V_plot = VoutData.Data;
                            t_plot = VoutData.Time;
                        else
                            V_plot = VoutData(:);
                            t_plot = simOut.tout;
                        end
                        
                        if isa(IoutData, 'timeseries')
                            I_plot = IoutData.Data;
                        else
                            I_plot = IoutData(:);
                        end
                        
                        % Create figure
                        fig = figure('Position', [100, 100, 1200, 600]);
                        
                        subplot(2, 1, 1);
                        plot(t_plot, V_plot, 'b-', 'LineWidth', 1.5);
                        grid on;
                        xlabel('Time (s)'); ylabel('Voltage (V)');
                        title(sprintf('%s - %s: Voltage (alpha = %d deg)', ...
                            strrep(modelBase, '_', ' '), strrep(scenario.name, '_', ' '), alpha_val));
                        
                        subplot(2, 1, 2);
                        plot(t_plot, I_plot, 'r-', 'LineWidth', 1.5);
                        grid on;
                        xlabel('Time (s)'); ylabel('Current (A)');
                        title(sprintf('Current (alpha = %d deg)', alpha_val));
                        
                        % Save figure
                        figName = sprintf('%s_%s_alpha_%d', modelBase, scenario.name, alpha_val);
                        saveas(fig, fullfile(figFolder, [figName '.png']));
                        saveas(fig, fullfile(figFolder, [figName '.fig']));
                        fprintf('  Saved graph for alpha=%d\n', alpha_val);
                        close(fig);
                    end
                end
            end
        end
        
        try
            close_system(modelPath,0);
        catch
        end
    end
end

% Save all results
save(fullfile(saveFolder,'all_results.mat'),'results','-v7.3');
fprintf('\nAll sweeps finished. Results in: %s\n', saveFolder);
\end{lstlisting}

\newpage
\section{Simulink Models}

The following Simulink models were developed for load analysis simulations:

\subsection{Rectifier Configuration Models}

\begin{enumerate}
    \item \texttt{center\_taped.slx} - Center-tapped full-wave rectifier with thyristor control
    \item \texttt{Full\_Wave\_Bridge.slx} - Full-wave bridge rectifier with four thyristors
    \item \texttt{half\_wave.slx} - Half-wave controlled rectifier with single thyristor
\end{enumerate}

\subsection{Diode Reference Models}

\begin{enumerate}
    \item \texttt{ct\_diode.slx} - Center-tapped rectifier with diodes (uncontrolled)
    \item \texttt{fw\_diode.slx} - Full-wave bridge with diodes (uncontrolled)
    \item \texttt{hf\_diode.slx} - Half-wave rectifier with diode (uncontrolled)
\end{enumerate}

All Simulink models are located in the \texttt{load\_analysis/simulink/} directory and implement various load scenarios including resistive-only, R-L, and highly inductive loads. The models support automated parameter sweeps through the MATLAB script interface for comprehensive analysis of rectifier performance across different firing angles and load conditions.

\newpage
\subsection{Simulink/Simscape Circuit Diagrams}

The following figures show the Simscape implementation of the controlled rectifier circuits used in the load analysis simulations.

\subsubsection{Center-Tapped Full-Wave Rectifier}

\begin{figure}[H]
    \centering
    \includegraphics[width=0.95\textwidth]{Load_Analysis/CT/CTSIMULINK.png}
    \caption{Center-tapped full-wave controlled rectifier Simscape model. The circuit uses two thyristors with phase-shifted gate signals to achieve full-wave rectification. A center-tapped transformer provides the dual voltage sources required for this topology.}
    \label{fig:ct_simulink}
\end{figure}

\subsubsection{Full-Wave Bridge Rectifier}

\begin{figure}[H]
    \centering
    \includegraphics[width=0.95\textwidth]{Load_Analysis/BG/bgSimulink.jpeg}
    \caption{Full-wave bridge controlled rectifier Simscape model. The circuit employs four thyristors arranged in a bridge configuration, allowing for full-wave rectification without requiring a center-tapped transformer. Gate signals are synchronized to control the firing angles of thyristor pairs.}
    \label{fig:bg_simulink}
\end{figure}

\subsubsection{Half-Wave Rectifier}

\begin{figure}[H]
    \centering
    \includegraphics[width=0.95\textwidth]{Load_Analysis/HF/hfSimulink.jpeg}
    \caption{Half-wave controlled rectifier Simscape model. This simplified topology uses a single thyristor to control current flow during the positive half-cycle of the AC input. The circuit demonstrates the basic principles of phase-controlled rectification.}
    \label{fig:hf_simulink}
\end{figure}

\end{document}
