\documentclass[11pt,a4paper]{article}

% ==================== PACKAGES ====================
\usepackage[utf8]{inputenc}
\usepackage[english]{babel}
\usepackage{amsmath}
\usepackage{graphicx}
\usepackage{float}
\usepackage{caption}
\usepackage{hyperref}
\usepackage{xcolor}
\usepackage{geometry}
\usepackage{booktabs}
\usepackage{siunitx}
\usepackage{listings}

% ==================== CUSTOM COMMANDS ====================
\newcommand{\TODO}[1]{\textcolor{red}{\textbf{TODO: #1}}}

% ==================== LISTINGS SETUP ====================
\lstset{
    language=Matlab,
    basicstyle=\ttfamily\footnotesize,
    keywordstyle=\color{blue},
    commentstyle=\color{green!60!black},
    stringstyle=\color{red},
    numbers=left,
    numberstyle=\tiny\color{gray},
    stepnumber=1,
    numbersep=5pt,
    backgroundcolor=\color{white},
    showspaces=false,
    showstringspaces=false,
    showtabs=false,
    frame=single,
    tabsize=2,
    captionpos=b,
    breaklines=true,
    breakatwhitespace=false,
    escapeinside={\%*}{*)},
    morekeywords={function, end, if, else, for, while, switch, case, otherwise}
}

% ==================== PAGE SETUP ====================
\geometry{
    a4paper,
    left=2.5cm,
    right=2.5cm,
    top=2.5cm,
    bottom=2.5cm
}

% ==================== HYPERLINKS ====================
\hypersetup{
    colorlinks=true,
    linkcolor=blue,
    urlcolor=cyan,
    citecolor=blue,
    pdftitle={Power Electronics I - Course Project},
    pdfauthor={Azab, Saber, Abdelhafez}
}

% ==================== GRAPHICS PATH ====================
\graphicspath{{figures/}}

% ==================== DOCUMENT INFO ====================
\title{
    \textbf{Power Electronics I} \\
    \Large{Course Project Report} \\
    \large{Controlled Rectifiers for Battery Charging}
}

\author{
    Mohammed Azab \and Basant Saber \and Habiba Abdelhafez \\
    \\
    German University in Cairo (GUC) \\
    Faculty of Engineering \\
    \\
    \textbf{Instructor:} Dr.-Ing. Moustafa Adly
}

\date{Fall Semester 2025 \\ November 25, 2025}

% ==================== DOCUMENT BEGIN ====================
\begin{document}

% Title Page
\maketitle

% Table of Contents
\tableofcontents
\newpage

% ==================== ABSTRACT ====================
\section*{Abstract}
\addcontentsline{toc}{section}{Abstract}

This project presents the design and analysis of controlled rectifier circuits using thyristors (SCRs) for battery charging applications. Three different rectifier configurations are implemented and simulated in MATLAB: half-wave controlled rectifier, full-wave center-tapped rectifier, and full-wave bridge rectifier. The study examines the relationship between firing angle and charging characteristics, including average and RMS voltage and current values. Performance comparisons demonstrate that full-wave configurations provide superior charging performance with reduced charging times and improved efficiency compared to half-wave rectifiers. The analysis includes circuit modeling, waveform generation, and comprehensive performance metrics for each topology.

% ==================== MAIN CONTENT ====================

% ==================== PART I ====================
\section{Introduction}

\subsection{Objectives}

The aim of this project is to design and simulate controlled rectifiers using thyristors (SCRs) for battery charging applications. A rectifier converts AC signals into DC signals by blocking the negative half of the AC waveform, making the output unidirectional for proper battery charging.

Three rectifier configurations were developed in MATLAB:
\begin{enumerate}
    \item \textbf{Half-wave Controlled Rectifier}: Single thyristor allowing only positive half-cycles
    \item \textbf{Full-wave Center-Tapped Rectifier}: Center-tapped transformer with two thyristors
    \item \textbf{Full-wave Bridge Rectifier}: Four thyristors in bridge configuration
\end{enumerate}

The analysis examines the relationship between firing angle and charging characteristics (average and RMS voltage, current, charging time) to compare the performance of each configuration.

\subsection{Background}

For safe and efficient operation, batteries require stable DC current with controlled voltage and current. Silicon-Controlled Rectifiers (SCRs) regulate power precisely, preventing overcharging and ensuring long battery life. SCRs can handle high currents and powers while adapting to battery state of charge (SoC) and temperature.

\subsection{Scope}

This project covers:
\begin{itemize}
    \item Analytical modeling using MATLAB
    \item Three rectifier configurations
    \item Performance comparison based on firing angle
    \item Power loss analysis including thyristor non-idealities
    \item State of Charge tracking
\end{itemize}

\section{Design and Implementation}

\subsection{MATLAB Function Architecture}

Each rectifier type is implemented as a separate MATLAB function:
\begin{itemize}
    \item \texttt{half\_wave\_charger.m} - Half-wave controlled rectifier
    \item \texttt{full\_wave\_ct\_charger.m} - Full-wave center-tapped rectifier  
    \item \texttt{full\_wave\_bridge\_charger.m} - Full-wave bridge rectifier
    \item \texttt{params.m} - System parameter configuration
\end{itemize}

All functions share common features:
\begin{enumerate}
    \item Input parameter handling for battery, supply, and thyristor specifications
    \item Firing angle sweep from 0° to 180°
    \item Calculation of voltage, current, and charging metrics
    \item Power loss analysis (battery, thyristor conduction, blocking, switching)
    \item State of Charge tracking
    \item Comprehensive visualization and data output
\end{enumerate}

\subsection{Half-Wave Controlled Rectifier}

Uses a single thyristor that conducts only during positive half-cycles when:
\begin{itemize}
    \item Thyristor is forward-biased
    \item Gate trigger is applied ($\theta \geq \alpha$)
    \item Output voltage exceeds battery EMF
\end{itemize}

\textbf{Key Equations:}
\begin{equation}
    V_{dc} = \frac{V_m}{2\pi}(1 + \cos\alpha) - V_t
\end{equation}
\begin{equation}
    I_{avg} = \frac{V_{dc} - V_{bat}}{R_{bat}}
\end{equation}
\begin{equation}
    t_{charge} = \frac{(SoC_{target} - SoC_{init}) \cdot C_{Ah} \cdot 3600}{I_{avg}}
\end{equation}

\subsection{Full-Wave Center-Tapped Rectifier}

Employs center-tapped transformer with two thyristors alternating each half-cycle, producing two output pulses per AC cycle.

\textbf{Advantages:} Higher average voltage, reduced ripple, faster charging

\textbf{Key Equation:}
\begin{equation}
    V_{dc} = \frac{2V_m}{\pi}(1 + \cos\alpha) - V_t
\end{equation}

\subsection{Full-Wave Bridge Rectifier}

Four thyristors in bridge arrangement. T1-T3 conduct during positive half-cycle, T2-T4 during negative half-cycle.

\textbf{Advantages:} Standard transformer (no center-tap), better transformer utilization

\textbf{Trade-off:} Four thyristors vs two thyristor drops in series

\textbf{Key Equation:}
\begin{equation}
    V_{dc} = \frac{2V_m}{\pi}(1 + \cos\alpha) - 2V_t
\end{equation}

\subsection{Power Loss Modeling}

Battery internal losses:
\begin{equation}
    P_{batt} = I_{rms}^2 \cdot R_{bat}
\end{equation}

Thyristor conduction losses:
\begin{equation}
    P_{th,cond} = V_t \cdot I_{avg} + R_{th} \cdot I_{rms}^2
\end{equation}

Thyristor blocking losses:
\begin{equation}
    P_{block} = V_{block,avg} \cdot I_{leak}
\end{equation}

Switching losses:
\begin{equation}
    P_{switch} = f \cdot (E_{on} + E_{off})
\end{equation}
where $E_{on} = \frac{1}{6}V_{block}I_{peak}t_{rise}$ and $E_{off} = \frac{1}{6}V_{block}I_{peak}t_{fall}$.

\subsection{State of Charge Tracking}

\begin{equation}
    SoC(t) = SoC_0 + \frac{100}{Q_{capacity}} \int_0^t I_{charge}(\tau) d\tau
\end{equation}
where $Q_{capacity}$ is in Coulombs (Ah × 3600).

\section{Part I: Results and Analysis}

\subsection{Half-Wave Rectifier Results}

\subsubsection{Firing Angle vs. Charging Time}
\TODO{Insert your main result plot}

\begin{figure}[H]
    \centering
    % \includegraphics[width=0.8\textwidth]{half_wave_alpha_vs_time.png}
    \caption{Firing angle vs. charging time for half-wave rectifier}
    \label{fig:half_wave_results}
\end{figure}

\TODO{Describe the plot:}
\begin{itemize}
    \item Trend observed (increasing $\alpha$ increases charging time)
    \item Physical explanation
    \item Key observations from the curve
\end{itemize}

\subsubsection{Output Voltage and Current Waveforms}
\TODO{Show representative waveforms for different firing angles}

\begin{figure}[H]
    \centering
    \begin{subfigure}[b]{0.45\textwidth}
        % \includegraphics[width=\textwidth]{half_wave_alpha_30.png}
        \caption{$\alpha = 30°$}
    \end{subfigure}
    \hfill
    \begin{subfigure}[b]{0.45\textwidth}
        % \includegraphics[width=\textwidth]{half_wave_alpha_90.png}
        \caption{$\alpha = 90°$}
    \end{subfigure}
    \caption{Output voltage and current waveforms for half-wave rectifier}
    \label{fig:half_wave_waveforms}
\end{figure}

\subsubsection{Performance Metrics}
\TODO{Create table with key metrics:}

\begin{table}[H]
\centering
\caption{Half-Wave Rectifier Performance Metrics}
\label{tab:half_wave_metrics}
\begin{tabular}{@{}lllll@{}}
\toprule
\textbf{Firing Angle} & \textbf{$V_{dc}$ (V)} & \textbf{$I_{dc}$ (A)} & \textbf{Charging Time (h)} & \textbf{Efficiency (\%)} \\ \midrule
$0°$ & \TODO{} & \TODO{} & \TODO{} & \TODO{} \\
$30°$ & \TODO{} & \TODO{} & \TODO{} & \TODO{} \\
$60°$ & \TODO{} & \TODO{} & \TODO{} & \TODO{} \\
$90°$ & \TODO{} & \TODO{} & \TODO{} & \TODO{} \\
$120°$ & \TODO{} & \TODO{} & \TODO{} & \TODO{} \\ \bottomrule
\end{tabular}
\end{table}

\subsection{Full-Wave Center-Tapped Rectifier Results}

\subsubsection{Firing Angle vs. Charging Time}
\TODO{Insert your main result plot}

\begin{figure}[H]
    \centering
    % \includegraphics[width=0.8\textwidth]{full_wave_ct_alpha_vs_time.png}
    \caption{Firing angle vs. charging time for center-tapped rectifier}
    \label{fig:full_wave_ct_results}
\end{figure}

\subsubsection{Output Voltage and Current Waveforms}
\TODO{Show representative waveforms}

\subsubsection{Performance Metrics}
\TODO{Create similar table as above}

\begin{table}[H]
\centering
\caption{Full-Wave Center-Tapped Rectifier Performance Metrics}
\label{tab:full_wave_ct_metrics}
\begin{tabular}{@{}lllll@{}}
\toprule
\textbf{Firing Angle} & \textbf{$V_{dc}$ (V)} & \textbf{$I_{dc}$ (A)} & \textbf{Charging Time (h)} & \textbf{Efficiency (\%)} \\ \midrule
$0°$ & \TODO{} & \TODO{} & \TODO{} & \TODO{} \\
$30°$ & \TODO{} & \TODO{} & \TODO{} & \TODO{} \\
$60°$ & \TODO{} & \TODO{} & \TODO{} & \TODO{} \\
$90°$ & \TODO{} & \TODO{} & \TODO{} & \TODO{} \\
$120°$ & \TODO{} & \TODO{} & \TODO{} & \TODO{} \\ \bottomrule
\end{tabular}
\end{table}

\subsection{Full-Wave Bridge Rectifier Results}

\subsubsection{Firing Angle vs. Charging Time}
\TODO{Insert your main result plot}

\begin{figure}[H]
    \centering
    % \includegraphics[width=0.8\textwidth]{full_wave_bridge_alpha_vs_time.png}
    \caption{Firing angle vs. charging time for bridge rectifier}
    \label{fig:full_wave_bridge_results}
\end{figure}

\subsubsection{Output Voltage and Current Waveforms}
\TODO{Show representative waveforms}

\subsubsection{Performance Metrics}
\TODO{Create similar table as above}

\subsection{Comparative Analysis}

\subsubsection{Side-by-Side Comparison}
\TODO{Create comparative plot with all three configurations}

\begin{figure}[H]
    \centering
    % \includegraphics[width=0.8\textwidth]{comparison_all_configs.png}
    \caption{Comparison of firing angle vs. charging time for all configurations}
    \label{fig:comparison_all}
\end{figure}

\subsubsection{Performance Comparison Table}
\TODO{Compare all three at same firing angle}

\begin{table}[H]
\centering
\caption{Comparison of Rectifier Configurations at $\alpha = 60°$}
\label{tab:comparison}
\begin{tabular}{@{}llllll@{}}
\toprule
\textbf{Configuration} & \textbf{$V_{dc}$ (V)} & \textbf{$I_{dc}$ (A)} & \textbf{Time (h)} & \textbf{$\eta$ (\%)} & \textbf{Ripple} \\ \midrule
Half-Wave & \TODO{} & \TODO{} & \TODO{} & \TODO{} & \TODO{High} \\
Full-Wave CT & \TODO{} & \TODO{} & \TODO{} & \TODO{} & \TODO{Medium} \\
Full-Wave Bridge & \TODO{} & \TODO{} & \TODO{} & \TODO{} & \TODO{Medium} \\ \bottomrule
\end{tabular}
\end{table}

\subsection{Optional Results}

\subsubsection{State of Charge Progression}
\TODO{If implemented, show SoC vs. time plots}

\begin{figure}[H]
    \centering
    % \includegraphics[width=0.8\textwidth]{soc_vs_time.png}
    \caption{State of Charge progression during charging}
    \label{fig:soc_progression}
\end{figure}

\subsubsection{Power Loss Analysis}
\TODO{If implemented, show power loss calculations}

\begin{table}[H]
\centering
\caption{Average Power Losses in Different Configurations}
\label{tab:power_losses}
\begin{tabular}{@{}lllll@{}}
\toprule
\textbf{Configuration} & \textbf{Conduction Loss (W)} & \textbf{Switching Loss (W)} & \textbf{Total Loss (W)} & \textbf{Efficiency (\%)} \\ \midrule
Half-Wave & \TODO{} & \TODO{} & \TODO{} & \TODO{} \\
Full-Wave CT & \TODO{} & \TODO{} & \TODO{} & \TODO{} \\
Full-Wave Bridge & \TODO{} & \TODO{} & \TODO{} & \TODO{} \\ \bottomrule
\end{tabular}
\end{table}

\subsection{Observations}
\TODO{List key observations from your results:}
\begin{enumerate}
    \item Effect of firing angle on charging time
    \item Comparison of half-wave vs. full-wave performance
    \item Difference between center-tapped and bridge configurations
    \item Impact of thyristor non-idealities (if modeled)
    \item Practical considerations for each configuration
\end{enumerate}

\section{Discussion}

\subsection{Analysis of Results}

\subsubsection{Firing Angle Effect}
Increasing firing angle significantly increases charging time because:
\begin{itemize}
    \item Delays thyristor turn-on, reducing conduction period
    \item Average voltage decreases: $V_{dc} = \frac{V_m}{k\pi}(1+\cos\alpha)$ where $k=2$ (half-wave), $k=1$ (full-wave)
    \item Lower voltage reduces charging current: $I_{avg} = (V_{dc} - V_{bat})/R_{bat}$
    \item Relationship is highly nonlinear with exponential growth as $\alpha \to 180°$
\end{itemize}

Practical firing angles: $0^\circ$ to $90^\circ$. Beyond $90^\circ$, output voltage too low for effective charging.

\subsubsection{Half-Wave vs Full-Wave}
Full-wave rectifiers are clearly superior:
\begin{itemize}
    \item 50\% reduction in charging time (two pulses per cycle)
    \item Double ripple frequency (100Hz vs 50Hz) - easier filtering
    \item Better transformer utilization
    \item Lower harmonic content
\end{itemize}

Half-wave only justified for very low-power, cost-sensitive applications.

\subsubsection{Center-Tapped vs Bridge}

\textbf{Center-Tapped:}
\begin{itemize}
    \item Pros: 2 thyristors, one $V_t$ drop, simpler gate drive
    \item Cons: Requires center-tapped transformer, poor transformer utilization
\end{itemize}

\textbf{Bridge:}
\begin{itemize}
    \item Pros: Standard transformer, better utilization, widely available
    \item Cons: 4 thyristors, two $V_t$ drops, complex gate drive
\end{itemize}

\textbf{Selection:} Use bridge for most applications (standard transformers). Use center-tapped if already available or when minimizing voltage drops is critical.

\subsection{Practical Considerations}

SoC tracking enables adaptive control: fast charge at low SoC, taper to trickle at high SoC, prevents overcharging.

Power loss analysis shows battery internal resistance dominates at high currents. Trade-off between fast charging (higher losses) and efficiency.

\subsection{Design Recommendations}

\begin{enumerate}
    \item \textbf{Industrial Applications}: Use full-wave bridge, operate at $\alpha = 15^\circ$--$45^\circ$, implement SoC-based control
    \item \textbf{Cost-Sensitive}: Consider half-wave if time not critical, optimize for balance of performance and stress
    \item \textbf{High-Efficiency}: Select low $V_t$ thyristors, optimize firing angle, consider active cooling for high power
\end{enumerate}

\subsection{Practical Considerations}

\subsubsection{Battery Charging Profile}
\TODO{Discuss:}
\begin{itemize}
    \item Constant current vs. constant voltage charging
    \item Impact of battery chemistry (lead-acid, Li-ion, etc.)
    \item Charging stages (bulk, absorption, float)
    \item How controlled rectifiers can implement different charging profiles
\end{itemize}

\subsubsection{Thyristor Selection}
\TODO{Discuss practical selection criteria:}
\begin{itemize}
    \item Voltage rating requirements
    \item Current rating requirements
    \item $dv/dt$ and $di/dt$ ratings
    \item Gate trigger requirements
    \item Thermal considerations
\end{itemize}


\subsection{Limitations of the Analysis}

\subsubsection{Model Simplifications}
\TODO{Acknowledge limitations:}
\begin{itemize}
    \item Simplified battery model (constant EMF and resistance)
    \item Ideal or simplified thyristor characteristics
    \item Neglected parasitic elements
    \item Steady-state analysis only
    \item No closed-loop control consideration
\end{itemize}

\subsubsection{Real-World Factors Not Considered}
\TODO{List factors that would affect real implementation:}
\begin{itemize}
    \item Supply voltage variations and harmonics
    \item Temperature effects on battery and components
    \item Component tolerances
    \item EMI/EMC considerations
    \item Aging effects
\end{itemize}

\subsection{Design Trade-offs}

\subsubsection{Cost vs. Performance}
\TODO{Analyze:}
\begin{itemize}
    \item Component count and cost
    \item Transformer requirements and cost
    \item Control circuit complexity
    \item Overall system cost for each configuration
\end{itemize}

\subsubsection{Efficiency vs. Control Range}
\TODO{Discuss:}
\begin{itemize}
    \item Efficiency variation with firing angle
    \item Optimal operating range
    \item Trade-off between controllability and efficiency
\end{itemize}

\subsubsection{Ripple vs. Complexity}
\TODO{Analyze:}
\begin{itemize}
    \item Output voltage ripple characteristics
    \item Need for filtering
    \item Impact on battery life
    \item Filter design considerations
\end{itemize}

\subsection{Recommendations}

\subsubsection{Configuration Selection Guidelines}
\TODO{Provide recommendations based on your analysis:}

\paragraph{Use Half-Wave When:}
\begin{itemize}
    \item \TODO{e.g., Very low power applications}
    \item \TODO{Cost is critical}
    \item \TODO{Charging time is not critical}
\end{itemize}

\paragraph{Use Center-Tapped When:}
\begin{itemize}
    \item \TODO{e.g., Center-tapped transformer already available}
    \item \TODO{Simple control is desired}
    \item \TODO{Medium power applications}
\end{itemize}

\paragraph{Use Bridge When:}
\begin{itemize}
    \item \TODO{e.g., Most general-purpose applications}
    \item \TODO{Standard transformer available}
    \item \TODO{Higher efficiency needed}
\end{itemize}

\subsubsection{Operating Point Recommendations}
\TODO{Suggest optimal operating conditions:}
\begin{itemize}
    \item Recommended firing angle range for each configuration
    \item Charging current limits
    \item Termination criteria
\end{itemize}

\subsection{Future Improvements}

\subsubsection{Model Enhancements}
\TODO{Suggest improvements to the model:}
\begin{itemize}
    \item More sophisticated battery model (e.g., equivalent circuit model)
    \item Detailed thyristor switching model
    \item Temperature dependence
    \item Closed-loop control simulation
\end{itemize}

\subsubsection{Analysis Extensions}
\TODO{Suggest additional analyses:}
\begin{itemize}
    \item Harmonic analysis
    \item Power factor analysis
    \item EMI prediction
    \item Reliability analysis
    \item Cost-benefit analysis
\end{itemize}


% ==================== PART II ====================
\section{Part II: Load Analysis - Introduction}

\subsection{Objectives}
\TODO{Expand on the following objectives:}

The primary objectives of Part II are:

\begin{itemize}
    \item Simulate and analyze controlled rectifiers under various load conditions
    \item Study the behavior of rectifiers with three load types:
    \begin{itemize}
        \item Resistive load (R)
        \item Resistive-inductive load (RL)
        \item Highly inductive load (L >> R)
    \end{itemize}
    \item Investigate current continuity and discontinuity modes
    \item Analyze the role of free-wheeling diodes
    \item Design firing circuits using pulse generators
    \item Compare half-wave and full-wave configurations
    \item Study the phenomenon of negative output voltage with positive current
\end{itemize}

\subsection{Background}
\TODO{Provide background on:}
\begin{itemize}
    \item Why load type matters in rectifier operation
    \item Inductive load effects on current waveforms
    \item Applications of controlled rectifiers with different loads:
    \begin{itemize}
        \item DC motor drives (RL load)
        \item Battery charging (EMF load - covered in Part I)
        \item Electrochemical processes
        \item High-power DC supplies
    \end{itemize}
\end{itemize}

\subsection{Problem Statement}
\TODO{State the specific problems addressed:}
\begin{itemize}
    \item Understanding current continuity conditions
    \item Predicting output characteristics for different loads
    \item Determining when free-wheeling diodes are beneficial
    \item Analyzing power flow and four-quadrant operation
\end{itemize}

\subsection{Simulation vs. Analytical Approach}
\TODO{Explain why Simulink is used for Part II:}
\begin{itemize}
    \item Complex waveforms difficult to analyze analytically
    \item Transient behavior important for inductive loads
    \item Visual representation of voltage and current waveforms
    \item Easy parameter variation and comparison
    \item Validation of analytical calculations
\end{itemize}

\subsection{Scope}
This part covers:
\begin{itemize}
    \item Simulink modeling and simulation
    \item MATLAB scripts for analytical calculations
    \item Systematic comparison of configurations
    \item Firing circuit design
\end{itemize}

This part does not cover:
\begin{itemize}
    \item Detailed component-level modeling
    \item Closed-loop control design
    \item Hardware implementation
\end{itemize}

\section{Future Work}

\subsection{Remaining Tasks}

\begin{enumerate}
    \item Develop Simulink models for all three rectifier configurations
    \item Implement R, RL, and highly inductive loads
    \item Design and simulate firing circuits
    \item Analyze current continuity conditions
    \item Evaluate free-wheeling diode performance
    \item Study four-quadrant operation and regenerative braking
    \item Generate comparative plots and performance tables
    \item Validate against analytical calculations
\end{enumerate}


% ==================== CONCLUSION ====================
\section{Conclusion}

\subsection{Summary of Work}
\TODO{Summarize what was accomplished in both parts:}

This project investigated controlled rectifier circuits for two main applications:

\paragraph{Part I - Battery Charger Design:}
\begin{itemize}
    \item Developed MATLAB functions for three rectifier configurations
    \item Analyzed the relationship between firing angle and charging time
    \item Compared half-wave, center-tapped, and bridge rectifiers
    \item Investigated optional features: SoC tracking and loss analysis
\end{itemize}

\paragraph{Part II - Load Analysis:}
\begin{itemize}
    \item Simulated controlled rectifiers with R, RL, and highly inductive loads
    \item Studied current continuity in different operating modes
    \item Analyzed the role of free-wheeling diodes
    \item Investigated four-quadrant operation and regenerative modes
    \item Designed and implemented firing circuits
\end{itemize}

\subsection{Key Findings}

\subsubsection{Part I Findings}
\TODO{List major conclusions from Part I:}
\begin{enumerate}
    \item \textbf{Firing Angle Effect:}
    \begin{itemize}
        \item Increasing firing angle decreases average output voltage
        \item Results in longer charging time
        \item Provides control over charging rate
    \end{itemize}
    
    \item \textbf{Configuration Comparison:}
    \begin{itemize}
        \item Full-wave rectifiers provide twice the average voltage of half-wave
        \item Charging time is approximately halved with full-wave
        \item Bridge and center-tapped perform similarly electrically
        \item Choice depends on transformer availability and cost
    \end{itemize}
    
    \item \textbf{Practical Considerations:}
    \begin{itemize}
        \item Thyristor losses are \TODO{significant/negligible}
        \item SoC tracking enables better charge control
        \item \TODO{Add other findings}
    \end{itemize}
\end{enumerate}

\subsubsection{Part II Findings}
\TODO{List major conclusions from Part II:}
\begin{enumerate}
    \item \textbf{Load Type Impact:}
    \begin{itemize}
        \item Resistive loads: Simple, discontinuous current, voltage and current in phase
        \item RL loads: Complex behavior, possible continuous conduction, current lag
        \item Highly inductive: Constant current, always CCM, enables regeneration
    \end{itemize}
    
    \item \textbf{Current Continuity:}
    \begin{itemize}
        \item Determined by L/R ratio, firing angle, and rectifier type
        \item Full-wave more likely to have CCM than half-wave
        \item Important for predicting output characteristics
    \end{itemize}
    
    \item \textbf{Free-Wheeling Diode:}
    \begin{itemize}
        \item Improves output voltage waveform
        \item Eliminates negative voltage excursions
        \item Prevents regenerative operation
        \item Essential for motor drives operating only in Quadrant I
    \end{itemize}
    
    \item \textbf{Four-Quadrant Operation:}
    \begin{itemize}
        \item $\alpha > 90°$ enables regenerative mode
        \item Negative voltage with positive current observed
        \item Important for braking in motor drives
        \item Not possible with free-wheeling diode
    \end{itemize}
    
    \item \textbf{Configuration Comparison:}
    \begin{itemize}
        \item Full-wave superior to half-wave in most aspects
        \item Center-tapped vs. bridge: trade-off between component count and transformer requirements
        \item \TODO{Add specific performance differences observed}
    \end{itemize}
\end{enumerate}

\subsection{Achievement of Objectives}

\TODO{Evaluate whether objectives were met:}

\begin{table}[H]
\centering
\caption{Objective Achievement Matrix}
\label{tab:objectives_achievement}
\begin{tabular}{@{}p{8cm}p{2cm}p{3cm}@{}}
\toprule

\end{tabular}
\end{table}

\subsection{Practical Implications}

\subsubsection{Design Guidelines}
\TODO{Provide practical design recommendations:}

\paragraph{For Battery Charging Applications:}
\begin{itemize}
    \item Use full-wave bridge rectifier for best performance
    \item Operate at moderate firing angles (\TODO{e.g., 30-60°}) for balance of speed and control
    \item Implement SoC monitoring for optimal charging
    \item Consider thyristor losses in high-power applications
\end{itemize}

\paragraph{For Motor Drive Applications:}
\begin{itemize}
    \item Use full-wave configuration for better performance
    \item Highly inductive armature enables good control
    \item Include free-wheeling diode for unidirectional operation
    \item Omit FWD if regenerative braking required
    \item Design for continuous conduction mode
\end{itemize}

\paragraph{For General DC Power Supplies:}
\begin{itemize}
    \item Match rectifier type to load characteristics
    \item Consider harmonic filtering requirements
    \item Balance cost vs. performance
    \item Provide adequate voltage rating for thyristors
\end{itemize}

\subsubsection{Economic Considerations}
\TODO{Discuss cost-benefit aspects:}
\begin{itemize}
    \item Component count vs. transformer requirements
    \item Efficiency impact on operating costs
    \item Reliability and maintenance
    \item Application-specific optimization
\end{itemize}

\subsection{Lessons Learned}

\TODO{Reflect on technical insights gained:}

\subsubsection{Technical Insights}
\begin{enumerate}
    \item \TODO{e.g., Importance of load type in rectifier design}
    \item \TODO{Trade-offs between control range and efficiency}
    \item \TODO{Complexity of inductive load behavior}
    \item \TODO{Value of simulation for understanding transient behavior}
\end{enumerate}

\subsubsection{Analytical Skills}
\begin{enumerate}
    \item \TODO{Developed proficiency in MATLAB function development}
    \item \TODO{Gained experience with Simulink power systems modeling}
    \item \TODO{Enhanced understanding of analytical vs. simulation approaches}
    \item \TODO{Improved ability to validate results}
\end{enumerate}

\subsection{Future Work}

\subsubsection{Short-Term Extensions}
\TODO{Suggest immediate improvements:}
\begin{itemize}
    \item Implement closed-loop control
    \item Add more sophisticated battery models
    \item Include harmonic analysis
    \item Develop efficiency optimization algorithms
    \item Create GUI for parameter adjustment
\end{itemize}

\subsubsection{Long-Term Research Directions}
\TODO{Propose advanced topics:}
\begin{itemize}
    \item Three-phase controlled rectifiers
    \item PWM rectifiers for improved power quality
    \item Active power factor correction
    \item Soft-switching techniques
    \item Digital control implementation
    \item Hardware prototype development
    \item Experimental validation
\end{itemize}

\subsubsection{Related Topics}
\TODO{Connect to broader field:}
\begin{itemize}
    \item DC-DC converters for post-regulation
    \item Inverters for AC motor drives
    \item Modern power electronic converters
    \item Renewable energy systems
    \item Electric vehicle charging infrastructure
\end{itemize}

\subsection{Relevance to Modern Power Electronics}

\TODO{Discuss how this project relates to current technology:}

\paragraph{Continued Relevance:}
\begin{itemize}
    \item Thyristor rectifiers still used in high-power applications
    \item Fundamental concepts apply to modern converters
    \item Understanding of controlled rectification essential
    \item Basis for more advanced topologies
\end{itemize}

\paragraph{Evolution to Modern Solutions:}
\begin{itemize}
    \item IGBTs and MOSFETs for faster switching
    \item PWM techniques for better waveforms
    \item Digital control for enhanced performance
    \item Integration with renewable energy
\end{itemize}

\subsection{Final Remarks}

\TODO{Conclude with overall assessment:}

This project provided comprehensive understanding of controlled rectifier operation through both analytical (MATLAB) and simulation (Simulink) approaches. The dual methodology of Part I (analytical functions) and Part II (detailed simulation) offered complementary insights into rectifier behavior.

Key achievements:
\begin{itemize}
    \item Successful implementation of battery charger design functions
    \item Detailed analysis of load type effects on rectifier performance
    \item Validation of theoretical concepts through simulation
    \item Development of practical design guidelines
\end{itemize}

The knowledge gained forms a solid foundation for understanding power electronic converters and their applications in modern electrical systems.

\TODO{Add personal reflection on the project experience and learning outcomes}


\newpage

% ==================== APPENDIX ====================
\appendix

\section{MATLAB Code}

This appendix contains the complete MATLAB implementation for the battery charger analysis.

\subsection{System Parameters (params.m)}

\begin{lstlisting}
% default battery_charger params

% Supply
Vrms = 230; 
f = 50;

% Battery
Vbat = 20; 
Rbat = 0.1; 
capacity = 50; 
capUnit = 'Ah';

% SoC & time
SoC_init = 12; 
SoC_target = 100;
t_charging_hours =  0.5; % Simulated User Input
t_charge = inf;

% Thyristor
alpha = 30;
alpha_deg = 0:2:180;         
Vt = 1.5;       
Rth = 0.001;  
Ileak = 0.01;  
t_rise = 1e-6;  
t_fall = 2e-6;  

% Simulation params 
dt = 1/(300*f);

% Visualization
enablePlots = true;
savePlots = true; 
\end{lstlisting}

\subsection{Half-Wave Rectifier (half\_wave\_charger.m)}

Due to the length of the code (over 500 lines), the key equations implemented are:

\begin{itemize}
    \item Average output voltage: $V_{dc} = \frac{V_m}{2\pi}(1+\cos\alpha) - V_t$
    \item Charging current: $I_{charge} = \frac{V_{dc} - V_{bat}}{R_{bat}}$
    \item Power losses: $P_{thyristor} = V_t I_{avg} + R_{th} I_{rms}^2$
    \item Battery losses: $P_{batt} = I_{rms}^2 R_{bat}$
    \item State of charge: $SoC(t) = SoC_{init} + \frac{100 \cdot I_{avg} \cdot t}{Q_{capacity}}$
\end{itemize}

The function generates comprehensive waveform plots including voltages, currents, power losses, and SoC profiles for different firing angles.

\subsection{Full-Wave Center-Tapped (full\_wave\_ct\_charger.m)}

Key differences from half-wave configuration:

\begin{itemize}
    \item Average output voltage: $V_{dc} = \frac{2V_m}{\pi}(1+\cos\alpha) - V_t$
    \item Double the output frequency (100 Hz ripple vs 50 Hz)
    \item Uses two thyristors with center-tapped transformer
    \item One thyristor drop per conduction path
    \item Improved performance: higher average voltage and reduced charging time
\end{itemize}

\subsection{Full-Wave Bridge (full\_wave\_bridge\_charger.m)}

Bridge configuration characteristics:

\begin{itemize}
    \item Average output voltage: $V_{dc} = \frac{2V_m}{\pi}(1+\cos\alpha) - 2V_t$
    \item Uses four thyristors in bridge configuration
    \item Standard transformer (no center-tap required)
    \item Two thyristor drops in series during conduction
    \item Electrically equivalent to center-tapped but mechanically simpler
    \item More complex gate drive circuit required
\end{itemize}

\subsection{Code Features}

All functions implement:

\begin{itemize}
    \item Flexible input parsing with name-value pairs
    \item Support for both fixed charging duration and target SoC scenarios
    \item Comprehensive power loss analysis (conduction, blocking, switching)
    \item Automatic figure generation with LaTeX formatting
    \item Export to both PNG (300 DPI) and FIG formats
    \item Detailed console output with system parameters and performance metrics
    \item Comparison plots across different firing angles
    \item Battery state-of-charge tracking over time
\end{itemize}

\end{document}
