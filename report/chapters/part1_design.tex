\section{Design and Implementation}

\subsection{MATLAB Function Architecture}

Each rectifier type is implemented as a separate MATLAB function:
\begin{itemize}
    \item \texttt{half\_wave\_charger.m} - Half-wave controlled rectifier
    \item \texttt{full\_wave\_ct\_charger.m} - Full-wave center-tapped rectifier  
    \item \texttt{full\_wave\_bridge\_charger.m} - Full-wave bridge rectifier
    \item \texttt{params.m} - System parameter configuration
\end{itemize}

All functions share common features:
\begin{enumerate}
    \item Input parameter handling for battery, supply, and thyristor specifications
    \item Firing angle sweep from 0° to 180°
    \item Calculation of voltage, current, and charging metrics
    \item Power loss analysis (battery, thyristor conduction, blocking, switching)
    \item State of Charge tracking
    \item Comprehensive visualization and data output
\end{enumerate}

\subsection{Half-Wave Controlled Rectifier}

Uses a single thyristor that conducts only during positive half-cycles when:
\begin{itemize}
    \item Thyristor is forward-biased
    \item Gate trigger is applied ($\theta \geq \alpha$)
    \item Output voltage exceeds battery EMF
\end{itemize}

\textbf{Key Equations:}
\begin{equation}
    V_{dc} = \frac{V_m}{2\pi}(1 + \cos\alpha) - V_t
\end{equation}
\begin{equation}
    I_{avg} = \frac{V_{dc} - V_{bat}}{R_{bat}}
\end{equation}
\begin{equation}
    t_{charge} = \frac{(SoC_{target} - SoC_{init}) \cdot C_{Ah} \cdot 3600}{I_{avg}}
\end{equation}

\subsection{Full-Wave Center-Tapped Rectifier}

Employs center-tapped transformer with two thyristors alternating each half-cycle, producing two output pulses per AC cycle.

\textbf{Advantages:} Higher average voltage, reduced ripple, faster charging

\textbf{Key Equation:}
\begin{equation}
    V_{dc} = \frac{2V_m}{\pi}(1 + \cos\alpha) - V_t
\end{equation}

\subsection{Full-Wave Bridge Rectifier}

Four thyristors in bridge arrangement. T1-T3 conduct during positive half-cycle, T2-T4 during negative half-cycle.

\textbf{Advantages:} Standard transformer (no center-tap), better transformer utilization

\textbf{Trade-off:} Four thyristors vs two thyristor drops in series

\textbf{Key Equation:}
\begin{equation}
    V_{dc} = \frac{2V_m}{\pi}(1 + \cos\alpha) - 2V_t
\end{equation}

\subsection{Power Loss Modeling}

Battery internal losses:
\begin{equation}
    P_{batt} = I_{rms}^2 \cdot R_{bat}
\end{equation}

Thyristor conduction losses:
\begin{equation}
    P_{th,cond} = V_t \cdot I_{avg} + R_{th} \cdot I_{rms}^2
\end{equation}

Thyristor blocking losses:
\begin{equation}
    P_{block} = V_{block,avg} \cdot I_{leak}
\end{equation}

Switching losses:
\begin{equation}
    P_{switch} = f \cdot (E_{on} + E_{off})
\end{equation}
where $E_{on} = \frac{1}{6}V_{block}I_{peak}t_{rise}$ and $E_{off} = \frac{1}{6}V_{block}I_{peak}t_{fall}$.

\subsection{State of Charge Tracking}

\begin{equation}
    SoC(t) = SoC_0 + \frac{100}{Q_{capacity}} \int_0^t I_{charge}(\tau) d\tau
\end{equation}
where $Q_{capacity}$ is in Coulombs (Ah × 3600).
