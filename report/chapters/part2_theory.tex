\section{Part II: Theoretical Background}

\subsection{Load Types and Their Effects}

\subsubsection{Resistive Load (R)}
\TODO{Explain resistive load characteristics:}

For a purely resistive load:
\begin{equation}
    v_o(t) = R \cdot i_o(t)
\end{equation}

\begin{itemize}
    \item Current and voltage are in phase
    \item Current is discontinuous (follows voltage)
    \item No energy storage
    \item Simple analysis
\end{itemize}

\subsubsection{Resistive-Inductive Load (RL)}
\TODO{Explain RL load characteristics:}

For an RL load:
\begin{equation}
    v_o(t) = R \cdot i_o(t) + L \frac{di_o(t)}{dt}
\end{equation}

\begin{itemize}
    \item Current lags voltage
    \item Inductor opposes current changes
    \item May have continuous or discontinuous current
    \item Extinction angle $\beta > \pi$ possible
\end{itemize}

\TODO{Derive current equation for RL load}

\subsubsection{Highly Inductive Load (L >> R)}
\TODO{Explain highly inductive load:}

For $L >> R$:
\begin{equation}
    v_o(t) \approx L \frac{di_o(t)}{dt}
\end{equation}

\begin{itemize}
    \item Current is approximately constant
    \item Continuous conduction mode
    \item Current ripple is small
    \item Acts as a current source
\end{itemize}

\subsection{Current Continuity Analysis}

\subsubsection{Continuous Conduction Mode (CCM)}
\TODO{Define and explain CCM:}
\begin{itemize}
    \item Current never reaches zero
    \item Occurs when $L$ is sufficiently large
    \item Simpler analysis
    \item Condition: $\tau = L/R$ is large compared to period
\end{itemize}

\subsubsection{Discontinuous Conduction Mode (DCM)}
\TODO{Define and explain DCM:}
\begin{itemize}
    \item Current reaches zero during the cycle
    \item Occurs with smaller $L$ or higher $\alpha$
    \item Extinction angle $\beta < \pi$ (half-wave) or $\beta < 2\pi$ (full-wave)
    \item More complex analysis
\end{itemize}

\subsubsection{Boundary Between CCM and DCM}
\TODO{Derive condition for continuous conduction}

Critical inductance:
\begin{equation}
    L_{crit} = \TODO{Derive based on firing angle and load resistance}
\end{equation}

\subsection{Controlled Rectifier with Different Loads}

\subsubsection{Half-Wave Rectifier with R Load}
\TODO{Analyze:}

Average output voltage:
\begin{equation}
    V_{dc} = \frac{V_m}{2\pi}(1 + \cos\alpha)
\end{equation}

Current flows only when $v_s > 0$ and thyristor is triggered.

\subsubsection{Half-Wave Rectifier with RL Load}
\TODO{Analyze:}

Current equation (during conduction):
\begin{equation}
    i_o(t) = \frac{V_m}{Z}\left[\sin(\omega t - \phi) - \sin(\alpha - \phi)e^{-(t-\alpha/\omega)/\tau}\right]
\end{equation}

where:
\begin{itemize}
    \item $Z = \sqrt{R^2 + (\omega L)^2}$
    \item $\phi = \tan^{-1}(\omega L / R)$
    \item $\tau = L / R$
\end{itemize}

Extinction angle $\beta$ found by solving $i_o(\beta) = 0$.

\subsubsection{Full-Wave Rectifier with RL Load}
\TODO{Analyze differences from half-wave:}
\begin{itemize}
    \item Two pulses per cycle
    \item Modified current equation
    \item Different extinction angle calculation
    \item Generally more likely to have continuous conduction
\end{itemize}

\subsection{Free-Wheeling Diode}

\subsubsection{Purpose and Function}
\TODO{Explain the role of free-wheeling diode:}

\begin{figure}[H]
    \centering
    % Circuit diagram with free-wheeling diode
    \caption{Controlled rectifier with free-wheeling diode}
    \label{fig:fwd_circuit}
\end{figure}

Functions:
\begin{itemize}
    \item Provides path for inductive load current when thyristor turns off
    \item Prevents negative output voltage
    \item Improves load voltage waveform
    \item Reduces harmonic content
\end{itemize}

\subsubsection{Effect on Output Voltage}
\TODO{Analyze:}

Without FWD:
\begin{itemize}
    \item Output voltage can go negative
    \item Current continues through thyristor even with negative voltage
    \item Average voltage may be lower
\end{itemize}

With FWD:
\begin{itemize}
    \item Output voltage clamped to zero
    \item Current free-wheels through diode
    \item Higher average voltage
    \item Better for motor drives
\end{itemize}

\subsection{Four-Quadrant Operation}

\subsubsection{Quadrant Definition}
\TODO{Explain the four quadrants:}

\begin{itemize}
    \item Quadrant I: $v_o > 0$, $i_o > 0$ (Forward motoring)
    \item Quadrant II: $v_o < 0$, $i_o > 0$ (Forward regeneration)
    \item Quadrant III: $v_o < 0$, $i_o < 0$ (Reverse motoring)
    \item Quadrant IV: $v_o > 0$, $i_o < 0$ (Reverse regeneration)
\end{itemize}

\subsubsection{Controlled Rectifier Capability}
\TODO{Explain which quadrants are accessible:}
\begin{itemize}
    \item Single controlled rectifier: Quadrants I and II
    \item With free-wheeling diode: Quadrant I only
    \item Dual converter: All four quadrants
\end{itemize}

\subsubsection{Negative Voltage with Positive Current}
\TODO{Explain this phenomenon:}

When $\alpha > 90°$ with inductive load:
\begin{itemize}
    \item Average voltage can be negative
    \item Current remains positive due to inductance
    \item Power flows from load to source (regeneration)
    \item Important for braking in motor drives
\end{itemize}

\subsection{Firing Circuit Design}

\subsubsection{Requirements}
\TODO{List firing circuit requirements:}
\begin{itemize}
    \item Synchronized with AC supply
    \item Adjustable firing angle
    \item Sufficient gate current pulse
    \item Isolation between power and control circuits
\end{itemize}

\subsubsection{Pulse Generator Approach}
\TODO{Explain how pulse generators are used in Simulink:}
\begin{itemize}
    \item Synchronization with AC source
    \item Phase delay adjustment
    \item Pulse width and amplitude
    \item Multiple channels for multiple thyristors
\end{itemize}

\subsection{Performance Metrics}

\subsubsection{Average Output Voltage}
\begin{equation}
    V_{dc} = \frac{1}{T} \int_0^T v_o(t) dt
\end{equation}

\subsubsection{RMS Output Voltage}
\begin{equation}
    V_{rms} = \sqrt{\frac{1}{T} \int_0^T v_o^2(t) dt}
\end{equation}

\subsubsection{Form Factor and Ripple Factor}
\TODO{Define and explain:}

Form Factor:
\begin{equation}
    FF = \frac{V_{rms}}{V_{dc}}
\end{equation}

Ripple Factor:
\begin{equation}
    RF = \sqrt{FF^2 - 1}
\end{equation}

\subsubsection{Power Factor}
\TODO{Explain power factor for controlled rectifiers:}
\begin{equation}
    PF = \frac{P_{avg}}{V_{s,rms} \cdot I_{s,rms}}
\end{equation}

Displacement Power Factor (DPF) and Distortion Factor.
