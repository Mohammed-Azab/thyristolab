\section{Conclusion}

\subsection{Summary of Work}

This project investigated controlled rectifier circuits for battery charging applications using thyristors (SCRs). The work was divided into two main parts:

\paragraph{Part I - Battery Charger Design:}
\begin{itemize}
    \item Developed MATLAB functions for three rectifier configurations: half-wave, full-wave center-tapped, and full-wave bridge
    \item Analyzed the relationship between firing angle and charging time
    \item Implemented State of Charge (SoC) tracking functionality
    \item Performed comprehensive power loss analysis including battery internal losses, thyristor conduction losses, blocking losses, and switching losses
    \item Generated extensive visualization of voltage and current waveforms, charging time curves, and power loss characteristics
\end{itemize}

\subsection{Final Remarks}

This project provided comprehensive understanding of controlled rectifier operation through both analytical (MATLAB) and simulation (Simulink) approaches. The dual methodology of Part I (analytical functions for battery charging) and Part II (detailed Simulink simulation of load effects) offered complementary insights into rectifier behavior under various operating conditions.

The study successfully demonstrated:
\begin{itemize}
    \item Implementation of battery charger design functions with SoC tracking and power loss analysis
    \item Detailed analysis of load type effects (resistive, R-L, highly inductive) on rectifier performance
    \item Validation of theoretical voltage and current equations through simulation
    \item Understanding of free-wheeling diode benefits for inductive loads
    \item Development of practical design guidelines for selecting rectifier configurations
\end{itemize}

The comprehensive MATLAB implementation and automated Simulink sweep analysis provide valuable tools for understanding the relationship between firing angle and charging performance across different rectifier topologies. The results clearly show that full-wave configurations are significantly superior to half-wave for practical applications, with the choice between bridge and center-tapped depending on transformer availability, cost considerations, and efficiency requirements.

The knowledge gained forms a solid foundation for understanding power electronic converters and their applications in modern electrical systems, bridging theoretical knowledge with practical engineering skills essential for power electronics design.
