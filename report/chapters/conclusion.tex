\section{Conclusion}

\subsection{Summary of Work}

This project investigated controlled rectifier circuits for battery charging applications using thyristors (SCRs). The work was divided into two main parts:

\paragraph{Part I - Battery Charger Design:}
\begin{itemize}
    \item Developed MATLAB functions for three rectifier configurations: half-wave, full-wave center-tapped, and full-wave bridge
    \item Analyzed the relationship between firing angle and charging time
    \item Implemented State of Charge (SoC) tracking functionality
    \item Performed comprehensive power loss analysis including battery internal losses, thyristor conduction losses, blocking losses, and switching losses
    \item Generated extensive visualization of voltage and current waveforms, charging time curves, and power loss characteristics
\end{itemize}

\paragraph{Part II - Load Analysis:}
\TODO{This section will be completed in the next phase of the project. Planned work includes:}
\begin{itemize}
    \item Simulink simulation of controlled rectifiers with R, RL, and highly inductive loads
    \item Study of current continuity in different operating modes
    \item Analysis of free-wheeling diode effects
    \item Investigation of four-quadrant operation and regenerative modes
    \item Firing circuit design and implementation
\end{itemize}

\subsection{Key Findings from Part I}

\begin{enumerate}
    \item \textbf{Firing Angle Effect:}
    \begin{itemize}
        \item Increasing firing angle decreases average output voltage and current
        \item Results in exponentially longer charging times
        \item Provides precise control over charging rate
        \item Enables adaptive charging strategies based on battery SoC
    \end{itemize}
    
    \item \textbf{Configuration Comparison:}
    \begin{itemize}
        \item Full-wave rectifiers provide approximately twice the average output voltage of half-wave
        \item Charging time is significantly reduced with full-wave configurations
        \item Bridge and center-tapped perform similarly in terms of electrical output
        \item Bridge configuration requires no center-tapped transformer but uses four thyristors
        \item Center-tapped uses only two thyristors but requires specialized transformer
    \end{itemize}
    
    \item \textbf{Performance Metrics:}
    \begin{itemize}
        \item Half-wave rectifier: Simplest but slowest charging, highest ripple
        \item Full-wave configurations: Better performance with reduced ripple and faster charging
        \item Power losses dominated by battery internal resistance at low firing angles
        \item Thyristor losses become more significant at higher current levels
    \end{itemize}
\end{enumerate}

\subsection{Final Remarks}

This project successfully demonstrated the design and analysis of thyristor-based controlled rectifiers for battery charging. The comprehensive MATLAB implementation provides a valuable tool for understanding the relationship between firing angle and charging performance across different rectifier topologies. The results show that full-wave configurations are significantly superior to half-wave for practical battery charging, with the choice between bridge and center-tapped depending on transformer availability and cost considerations.

The modular design of the MATLAB functions and the extensive visualization capabilities make this work a useful foundation for further research and educational applications in power electronics.

\subsection{Key Findings}

\subsubsection{Part I Findings}
\TODO{List major conclusions from Part I:}
\begin{enumerate}
    \item \textbf{Firing Angle Effect:}
    \begin{itemize}
        \item Increasing firing angle decreases average output voltage
        \item Results in longer charging time
        \item Provides control over charging rate
    \end{itemize}
    
    \item \textbf{Configuration Comparison:}
    \begin{itemize}
        \item Full-wave rectifiers provide twice the average voltage of half-wave
        \item Charging time is approximately halved with full-wave
        \item Bridge and center-tapped perform similarly electrically
        \item Choice depends on transformer availability and cost
    \end{itemize}
    
    \item \textbf{Practical Considerations:}
    \begin{itemize}
        \item Thyristor losses are \TODO{significant/negligible}
        \item SoC tracking enables better charge control
        \item \TODO{Add other findings}
    \end{itemize}
\end{enumerate}

\subsubsection{Part II Findings}
\TODO{List major conclusions from Part II:}
\begin{enumerate}
    \item \textbf{Load Type Impact:}
    \begin{itemize}
        \item Resistive loads: Simple, discontinuous current, voltage and current in phase
        \item RL loads: Complex behavior, possible continuous conduction, current lag
        \item Highly inductive: Constant current, always CCM, enables regeneration
    \end{itemize}
    
    \item \textbf{Current Continuity:}
    \begin{itemize}
        \item Determined by L/R ratio, firing angle, and rectifier type
        \item Full-wave more likely to have CCM than half-wave
        \item Important for predicting output characteristics
    \end{itemize}
    
    \item \textbf{Free-Wheeling Diode:}
    \begin{itemize}
        \item Improves output voltage waveform
        \item Eliminates negative voltage excursions
        \item Prevents regenerative operation
        \item Essential for motor drives operating only in Quadrant I
    \end{itemize}
    
    \item \textbf{Four-Quadrant Operation:}
    \begin{itemize}
        \item $\alpha > 90°$ enables regenerative mode
        \item Negative voltage with positive current observed
        \item Important for braking in motor drives
        \item Not possible with free-wheeling diode
    \end{itemize}
    
\end{enumerate}

\subsection{Final Remarks}

This project provided comprehensive understanding of controlled rectifier operation through both analytical (MATLAB) and simulation (Simulink) approaches. The dual methodology of Part I (analytical functions) and Part II (detailed simulation) offered complementary insights into rectifier behavior.

Key achievements:
\begin{itemize}
    \item Successful implementation of battery charger design functions
    \item Detailed analysis of load type effects on rectifier performance
    \item Validation of theoretical concepts through simulation
    \item Development of practical design guidelines
\end{itemize}

The knowledge gained forms a solid foundation for understanding power electronic converters and their applications in modern electrical systems.

\TODO{Add personal reflection on the project experience and learning outcomes}
