\section{Conclusion}

\subsection{Summary of Work}

This project investigated controlled rectifier circuits for battery charging applications using thyristors (SCRs). The work was divided into two main parts:

\paragraph{Part I - Battery Charger Design:}
\begin{itemize}
    \item Developed MATLAB functions for three rectifier configurations: half-wave, full-wave center-tapped, and full-wave bridge
    \item Analyzed the relationship between firing angle and charging time
    \item Implemented State of Charge (SoC) tracking functionality
    \item Performed comprehensive power loss analysis including battery internal losses, thyristor conduction losses, blocking losses, and switching losses
    \item Generated extensive visualization of voltage and current waveforms, charging time curves, and power loss characteristics
\end{itemize}

\subsection{Final Remarks}

This project successfully demonstrated the design and analysis of thyristor-based controlled rectifiers for battery charging. The comprehensive MATLAB implementation provides a valuable tool for understanding the relationship between firing angle and charging performance across different rectifier topologies. The results show that full-wave configurations are significantly superior to half-wave for practical battery charging, with the choice between bridge and center-tapped depending on transformer availability and cost considerations.

The modular design of the MATLAB functions and the extensive visualization capabilities make this work a useful foundation for further research and educational applications in power electronics.

\subsection{Final Remarks}

This project provided comprehensive understanding of controlled rectifier operation through both analytical (MATLAB) and simulation (Simulink) approaches. The dual methodology of Part I (analytical functions) and Part II (detailed simulation) offered complementary insights into rectifier behavior.

Key achievements:
\begin{itemize}
    \item Successful implementation of battery charger design functions
    \item Detailed analysis of load type effects on rectifier performance
    \item Validation of theoretical concepts through simulation
    \item Development of practical design guidelines
\end{itemize}

The knowledge gained forms a solid foundation for understanding power electronic converters and their applications in modern electrical systems.

\subsection{Reflections}

This project provided invaluable hands-on experience in power electronics design and simulation. Through implementing controlled rectifiers in MATLAB and analyzing their performance, we gained deep insights into thyristor operation, firing angle control, and practical considerations in power converter design. The comprehensive analysis of different rectifier topologies enhanced our understanding of trade-offs between circuit complexity, component count, and performance characteristics. This work bridges theoretical knowledge with practical engineering skills essential for modern power electronics applications.
