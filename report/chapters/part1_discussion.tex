\section{Discussion}

\subsection{Analysis of Results}

\subsubsection{Firing Angle Effect}
Increasing firing angle significantly increases charging time because:
\begin{itemize}
    \item Delays thyristor turn-on, reducing conduction period
    \item Average voltage decreases: $V_{dc} = \frac{V_m}{k\pi}(1+\cos\alpha)$ where $k=2$ (half-wave), $k=1$ (full-wave)
    \item Lower voltage reduces charging current: $I_{avg} = (V_{dc} - V_{bat})/R_{bat}$
    \item Relationship is highly nonlinear with exponential growth as $\alpha \to 180°$
\end{itemize}

Practical firing angles: $0^\circ$ to $90^\circ$. Beyond $90^\circ$, output voltage too low for effective charging.

\subsubsection{Half-Wave vs Full-Wave}
Full-wave rectifiers are clearly superior:
\begin{itemize}
    \item 50\% reduction in charging time (two pulses per cycle)
    \item Double ripple frequency (100Hz vs 50Hz) - easier filtering
    \item Better transformer utilization
    \item Lower harmonic content
\end{itemize}

Half-wave only justified for very low-power, cost-sensitive applications.

\subsubsection{Center-Tapped vs Bridge}

\textbf{Center-Tapped:}
\begin{itemize}
    \item Pros: 2 thyristors, one $V_t$ drop, simpler gate drive
    \item Cons: Requires center-tapped transformer, poor transformer utilization
\end{itemize}

\textbf{Bridge:}
\begin{itemize}
    \item Pros: Standard transformer, better utilization, widely available
    \item Cons: 4 thyristors, two $V_t$ drops, complex gate drive
\end{itemize}

\textbf{Selection:} Use bridge for most applications (standard transformers). Use center-tapped if already available or when minimizing voltage drops is critical.
