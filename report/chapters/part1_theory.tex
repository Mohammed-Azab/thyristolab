\section{Part I: Theoretical Background}

\subsection{Thyristor (SCR) Operation}
\TODO{Explain thyristor operation including:}
\begin{itemize}
    \item Structure and symbol
    \item Operating characteristics (forward blocking, reverse blocking, conduction)
    \item Triggering mechanism and gate requirements
    \item Turn-on and turn-off processes
    \item V-I characteristics
\end{itemize}

\TODO{Add figure showing thyristor symbol and V-I characteristics}
\begin{figure}[H]
    \centering
    % \includegraphics[width=0.7\textwidth]{thyristor_symbol.png}
    \caption{Thyristor symbol and characteristics}
    \label{fig:thyristor}
\end{figure}

\subsection{Half-Wave Controlled Rectifier}
\TODO{Describe half-wave rectifier theory:}

\subsubsection{Circuit Configuration}
\TODO{Add circuit diagram}
\begin{figure}[H]
    \centering
    % \begin{circuitikz}
    % % Draw circuit here
    % \end{circuitikz}
    \caption{Half-wave controlled rectifier circuit for battery charging}
    \label{fig:half_wave_circuit}
\end{figure}

\subsubsection{Mathematical Analysis}
\TODO{Derive the following equations:}

Input voltage:
\begin{equation}
    v_s(t) = V_m \sin(\omega t)
\end{equation}

where $V_m = \sqrt{2} \cdot V_{rms}$ and $\omega = 2\pi f$.

Average output voltage:
\begin{equation}
    V_{dc} = \frac{V_m}{2\pi} (1 + \cos\alpha)
    \label{eq:vdc_half_wave}
\end{equation}

RMS output voltage:
\begin{equation}
    V_{rms} = \frac{V_m}{2}\sqrt{\frac{1}{\pi}\left[(\pi - \alpha) + \frac{\sin(2\alpha)}{2}\right]}
    \label{eq:vrms_half_wave}
\end{equation}

where $\alpha$ is the firing angle.

\TODO{Derive the output current equation considering battery EMF and resistance}

\subsection{Full-Wave Controlled Rectifier - Center-Tapped}
\TODO{Describe center-tapped full-wave rectifier theory:}

\subsubsection{Circuit Configuration}
\TODO{Add circuit diagram with center-tapped transformer and two SCRs}
\begin{figure}[H]
    \centering
    % Circuit diagram
    \caption{Full-wave center-tapped controlled rectifier}
    \label{fig:full_wave_ct_circuit}
\end{figure}

\subsubsection{Mathematical Analysis}
\TODO{Derive the following:}

Average output voltage:
\begin{equation}
    V_{dc} = \frac{V_m}{\pi} (1 + \cos\alpha)
    \label{eq:vdc_full_wave_ct}
\end{equation}

RMS output voltage:
\begin{equation}
    V_{rms} = \frac{V_m}{\sqrt{2}}\sqrt{\frac{1}{\pi}\left[(\pi - \alpha) + \frac{\sin(2\alpha)}{2}\right]}
    \label{eq:vrms_full_wave_ct}
\end{equation}

\subsection{Full-Wave Bridge Rectifier}
\TODO{Describe bridge rectifier theory:}

\subsubsection{Circuit Configuration}
\TODO{Add circuit diagram with 4 SCRs in bridge configuration}
\begin{figure}[H]
    \centering
    % Circuit diagram
    \caption{Full-wave bridge controlled rectifier}
    \label{fig:full_wave_bridge_circuit}
\end{figure}

\subsubsection{Mathematical Analysis}
\TODO{Note that equations are same as center-tapped but circuit differences:}
\begin{itemize}
    \item No center-tapped transformer required
    \item Four thyristors instead of two
    \item Better transformer utilization
\end{itemize}

\subsection{Battery Charging Characteristics}
\TODO{Explain battery models:}

\subsubsection{Simplified Battery Model}
The battery is modeled as:
\begin{equation}
    v_{batt}(t) = E_{batt} + i_{batt}(t) \cdot R_{batt}
\end{equation}

where:
\begin{itemize}
    \item $E_{batt}$ is the battery EMF (depends on SoC)
    \item $R_{batt}$ is the internal resistance
    \item $i_{batt}(t)$ is the charging current
\end{itemize}

\subsubsection{State of Charge (SoC)}
\TODO{Explain SoC calculation:}

\begin{equation}
    SoC(t) = SoC_0 + \frac{1}{Q_{capacity}} \int_0^t i_{batt}(\tau) d\tau
\end{equation}

where $Q_{capacity}$ is the battery capacity in Ampere-hours (Ah).

\subsection{Power and Efficiency Analysis}
\TODO{Derive expressions for:}

Average power delivered to battery:
\begin{equation}
    P_{avg} = V_{dc} \cdot I_{dc}
\end{equation}

Power losses in thyristors (if considering non-idealities):
\begin{equation}
    P_{loss} = \TODO{Derive based on forward voltage drop and conduction time}
\end{equation}

Efficiency:
\begin{equation}
    \eta = \frac{P_{battery}}{P_{input}} \times 100\%
\end{equation}
