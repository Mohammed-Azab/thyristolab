\section{Introduction}

\subsection{Objectives}

The aim of this project is to design and simulate controlled rectifiers using thyristors (SCRs) for battery charging applications. A rectifier converts AC signals into DC signals by blocking the negative half of the AC waveform, making the output unidirectional for proper battery charging.

Three rectifier configurations were developed in MATLAB:
\begin{enumerate}
    \item \textbf{Half-wave Controlled Rectifier}: Single thyristor allowing only positive half-cycles
    \item \textbf{Full-wave Center-Tapped Rectifier}: Center-tapped transformer with two thyristors
    \item \textbf{Full-wave Bridge Rectifier}: Four thyristors in bridge configuration
\end{enumerate}

The analysis examines the relationship between firing angle and charging characteristics (average and RMS voltage, current, charging time) to compare the performance of each configuration.

\subsection{Background}

For safe and efficient operation, batteries require stable DC current with controlled voltage and current. Silicon-Controlled Rectifiers (SCRs) regulate power precisely, preventing overcharging and ensuring long battery life. SCRs can handle high currents and powers while adapting to battery state of charge (SoC) and temperature.

\subsection{Scope}

This project covers:
\begin{itemize}
    \item Analytical modeling using MATLAB
    \item Three rectifier configurations
    \item Performance comparison based on firing angle
    \item Power loss analysis including thyristor non-idealities
    \item State of Charge tracking
\end{itemize}
