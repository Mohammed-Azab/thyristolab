\section{Part I: Battery Charger Design - Introduction}

\subsection{Objectives}
% Expand on the following objectives with more detail:

The primary objectives of Part I are:

\begin{itemize}
    \item Design and simulate controlled rectifier circuits for battery charging applications
    \item Develop MATLAB functions for three different rectifier configurations:
    \begin{itemize}
        \item Half-wave controlled rectifier
        \item Full-wave controlled rectifier using center-tapped transformer
        \item Full-wave controlled rectifier using bridge configuration
    \end{itemize}
    \item Analyze the relationship between firing angle and charging characteristics
    \item Compare the performance of different rectifier topologies
    \item Investigate the impact of thyristor non-idealities on system performance
\end{itemize}

\subsection{Background}
% Provide background information on:
\begin{itemize}
    \item Battery charging requirements and characteristics
    \item Why controlled rectifiers are used for battery charging
    \item Advantages of phase-controlled rectifiers over uncontrolled ones
    \item Applications in electric vehicles, renewable energy systems, UPS systems
\end{itemize}

\subsection{Problem Statement}
% Clearly state the problem this part addresses:
\begin{itemize}
    \item Need for adjustable charging current/voltage
    \item Importance of State of Charge (SoC) monitoring
    \item Power efficiency considerations
    \item Design trade-offs between different configurations
\end{itemize}

\subsection{Scope}
% Define what is covered and what is outside the scope:

This part covers:
\begin{itemize}
    \item Analytical modeling using MATLAB
    \item Three rectifier configurations
    \item Performance comparison based on firing angle
    \item Optional analysis of thyristor losses
\end{itemize}

This part does not cover:
\begin{itemize}
    \item Detailed battery chemistry models
    \item Thermal analysis
    \item Control system design
\end{itemize}
