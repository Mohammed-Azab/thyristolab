\section{Part II: Simulation and Methodology}

\subsection{Simulation Parameters}

\subsubsection{Supply Specifications}
\TODO{List the supply parameters used:}

\begin{table}[H]
\centering
\caption{Supply Parameters}
\label{tab:supply_params}
\begin{tabular}{@{}lll@{}}
\toprule
\textbf{Parameter} & \textbf{Value} & \textbf{Unit} \\ \midrule
Supply Voltage (RMS) & 230 & V \\
Frequency & 50 & Hz \\
Phase & Single-phase & - \\ \bottomrule
\end{tabular}
\end{table}

\subsubsection{Load Specifications}
\TODO{Define the load parameters for each load type:}

\begin{table}[H]
\centering
\caption{Load Parameters}
\label{tab:load_params}
\begin{tabular}{@{}llll@{}}
\toprule
\textbf{Load Type} & \textbf{R ($\Omega$)} & \textbf{L (H)} & \textbf{L/R (ms)} \\ \midrule
Resistive & \TODO{e.g., 10} & 0 & 0 \\
RL Load & \TODO{e.g., 10} & \TODO{e.g., 0.05} & \TODO{e.g., 5} \\
Highly Inductive & \TODO{e.g., 5} & \TODO{e.g., 0.5} & \TODO{e.g., 100} \\ \bottomrule
\end{tabular}
\end{table}

\subsubsection{Thyristor Parameters}
\TODO{Specify thyristor parameters used in simulation:}

\begin{table}[H]
\centering
\caption{Thyristor Parameters}
\label{tab:thyristor_params}
\begin{tabular}{@{}lll@{}}
\toprule
\textbf{Parameter} & \textbf{Value} & \textbf{Unit} \\ \midrule
Forward voltage drop & \TODO{e.g., 1.5} & V \\
On-state resistance & \TODO{e.g., 0.01} & $\Omega$ \\
Snubber resistance & \TODO{e.g., 100} & $\Omega$ \\
Snubber capacitance & \TODO{e.g., 0.1} & $\mu$F \\ \bottomrule
\end{tabular}
\end{table}

\subsection{Simulink Model Development}

\subsubsection{Half-Wave Controlled Rectifier Model}

\paragraph{Circuit Components:}
\TODO{List and describe each component in your Simulink model:}
\begin{itemize}
    \item AC Voltage Source (230V RMS, 50 Hz)
    \item Thyristor (from Simscape Electrical)
    \item Load (R, RL, or highly inductive)
    \item Firing circuit (Pulse Generator)
    \item Free-wheeling diode (optional, for comparison)
    \item Measurement blocks (Voltage Sensor, Current Sensor)
    \item Scope blocks for visualization
\end{itemize}

\paragraph{Block Diagram:}
\TODO{Insert screenshot or diagram of your Simulink model}

\begin{figure}[H]
    \centering
    % \includegraphics[width=0.9\textwidth]{half_wave_simulink.png}
    \caption{Half-wave controlled rectifier Simulink model}
    \label{fig:half_wave_simulink}
\end{figure}

\paragraph{Firing Circuit Design:}
\TODO{Explain your firing circuit implementation:}
\begin{itemize}
    \item Synchronization method with AC source
    \item Firing angle adjustment mechanism
    \item Pulse width selection
    \item Pulse amplitude (gate current)
\end{itemize}

\begin{figure}[H]
    \centering
    % \includegraphics[width=0.7\textwidth]{firing_circuit.png}
    \caption{Firing circuit implementation}
    \label{fig:firing_circuit}
\end{figure}

\subsubsection{Full-Wave Center-Tapped Rectifier Model}

\paragraph{Circuit Modifications:}
\TODO{Explain differences from half-wave model:}
\begin{itemize}
    \item Center-tapped transformer implementation
    \item Two thyristors with separate firing circuits
    \item Phase shift between firing pulses (180°)
    \item Load connection
\end{itemize}

\paragraph{Block Diagram:}
\TODO{Insert screenshot}

\begin{figure}[H]
    \centering
    % \includegraphics[width=0.9\textwidth]{full_wave_ct_simulink.png}
    \caption{Full-wave center-tapped rectifier Simulink model}
    \label{fig:full_wave_ct_simulink}
\end{figure}

\subsubsection{Full-Wave Bridge Rectifier Model}

\paragraph{Circuit Components:}
\TODO{Describe bridge configuration:}
\begin{itemize}
    \item Four thyristors
    \item Firing signal pairing (T1-T3 and T2-T4)
    \item No center-tapped transformer needed
\end{itemize}

\paragraph{Block Diagram:}
\TODO{Insert screenshot}

\begin{figure}[H]
    \centering
    % \includegraphics[width=0.9\textwidth]{full_wave_bridge_simulink.png}
    \caption{Full-wave bridge rectifier Simulink model}
    \label{fig:full_wave_bridge_simulink}
\end{figure}

\subsection{Simulation Setup}

\subsubsection{Solver Configuration}
\TODO{Document solver settings:}
\begin{itemize}
    \item Solver type: \TODO{e.g., ode23tb (stiff/TR-BDF2)}
    \item Max step size: \TODO{e.g., auto or 1e-5}
    \item Relative tolerance: \TODO{e.g., 1e-3}
    \item Absolute tolerance: \TODO{e.g., 1e-6}
    \item Simulation time: \TODO{e.g., 0.1 sec (5 cycles)}
\end{itemize}

\subsubsection{Measurement and Visualization}
\TODO{Describe how measurements are taken:}
\begin{itemize}
    \item Voltage measurement points
    \item Current measurement points
    \item Scope configuration for waveform display
    \item Data export to MATLAB workspace
\end{itemize}

\subsection{Analytical Calculation Scripts}

\subsubsection{Purpose}
\TODO{Explain what the analytical scripts calculate:}
\begin{itemize}
    \item Theoretical average and RMS values
    \item Comparison with simulation results
    \item Validation of simulation accuracy
\end{itemize}

\subsubsection{Script Structure}
\TODO{Outline your MATLAB analytical scripts:}

\begin{lstlisting}
% Example structure
function [Vdc, Vrms, Idc, Irms] = analytical_half_wave_R(Vs, f, alpha, R)
    % Calculate theoretical values
    % ...
end
\end{lstlisting}

\subsection{Experimental Procedure}

\subsubsection{Test Cases}
\TODO{Define your systematic test procedure:}

\paragraph{Firing Angle Sweep:}
Test firing angles: $\alpha = $ \TODO{e.g., 0°, 30°, 60°, 90°, 120°, 150°}

\paragraph{Load Type Variation:}
For each firing angle, test:
\begin{enumerate}
    \item Resistive load
    \item RL load
    \item Highly inductive load
\end{enumerate}

\paragraph{Configuration Comparison:}
For each load type and firing angle:
\begin{enumerate}
    \item Half-wave rectifier
    \item Full-wave center-tapped rectifier
    \item Full-wave bridge rectifier
\end{enumerate}

\paragraph{Free-Wheeling Diode Study:}
For RL loads, compare:
\begin{enumerate}
    \item Without free-wheeling diode
    \item With free-wheeling diode
\end{enumerate}

\subsubsection{Data Collection}
\TODO{Describe what data you collect for each test:}
\begin{itemize}
    \item Output voltage waveform
    \item Output current waveform
    \item Thyristor voltage waveform
    \item Thyristor current waveform
    \item Average output voltage
    \item RMS output voltage
    \item Average output current
    \item RMS output current
    \item Extinction angle (for discontinuous mode)
\end{itemize}

\subsection{Validation}

\subsubsection{Comparison with Analytical Results}
\TODO{Describe validation procedure:}
\begin{itemize}
    \item Calculate theoretical values using derived equations
    \item Compare with simulation results
    \item Calculate percentage error
    \item Acceptable error threshold: \TODO{e.g., < 5\%}
\end{itemize}

\subsubsection{Physical Plausibility Checks}
\TODO{List checks performed:}
\begin{itemize}
    \item Current never negative for unidirectional devices
    \item Average voltage decreases with increasing firing angle
    \item Energy conservation (for ideal components)
    \item Correct phase relationships
\end{itemize}
